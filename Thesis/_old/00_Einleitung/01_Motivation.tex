\section{Motivation der Arbeit}

Moderne Fahrzeuge wandeln sich immer mehr zum komplexen Netzwerken hin. Durch die Globalisierung und Urbanisierung, wächst das Verkehrsaufkommen weltweit schnell an. Mit diesem Mehraufwand an Fahrzeugen wird das aktuelle Verkehrsnetz an seine Kapazitätsgrenze kommen. Laut einer aktuellen Prognose, sollen im Jahr 2050 bereits 70 Prozent der Weltbevölkerung in Städten leben. Dementsprechenden wird die Zahl der Automobile in diesen Ballungsgebieten stark zunehmen \cite{Urbanisierung}. 

Die geplante Automatisierung und Vernetzung von Fahrzeugen wird in diesen Gebieten ein große Herausforderung darstellen. Sie erschließen aber auch neue Möglichkeiten, die das Autofahren effizienter, sicherer und umweltfreundlicher machen sollen. Da deutsche Automobilhersteller und Zulieferer in diesen Bereichen bereits seit mehre Jahre tätig sind, wollen dies ihre Innovationsführerschaft weiter ausbauen. So verkündete der deutsche Automobilkonzern BMW und der chinesische Automobilhersteller Great Wall ihre Zusammenarbeit bei der Entwicklung selbstfahrender Autos und der Entwicklung gemeinsamer Standards \cite{BMW}.

Auch die Bundesrepublik Deutschland sieht sich in einer Vorreiter Rolle zu den Thema autonomes Fahren und hat sich, wie mehrfach in den Medien bekannt gegeben, hierzu einige Strategien überlegt. So soll in den nächsten Jahren ein Eingliederung des autonomen Fahren in den Straßenverkehr schrittweise erfolgen. Zu diesem Thema hat das Bundesministerium für Verkehr und digitale Infrastruktur, Handlungsempfehlungen zu Probebetrieb, Entwicklung zur Serienreife und Zulassungen erstellt \cite{BMVI}.

Assistenzsysteme, die den Fahrer bei der Führung eins Automobils unterstützen, sind heute bereits der Regelfall in modernen Fahrzeugen. Die Anzahl dieser Systeme soll in den nächsten Jahren weiter ansteigen, bis diese ein vollständig autonomes Fahren eines Fahrzeuges ermöglichen. Das autonome Fahren wird sukzessive weiterentwickelt und mehr und mehr in die nächsten Generationen von Autos integriert werden. So wird sich das selbstständige Fahren Schritt für Schritt in den Straßenverkehr etablieren.  

Der Plan ist es das automatisierte Fahren zuerst auf Autobahnen und abgeschlossenen Bereichen, wie Parkanlagen, zu testen. Letzteres bietet sich an, da hier geringe Geschwindigkeiten auf komplexe Situationen aus dem Straßenverkehr treffen. Hierbei muss ein Fahrzeug sein komplettes Umfeld selbstständig mit den erwähnten Assistenzsystemen überwachen und jede Situation damit fehlerfrei beherrschen. Im Bereich autonomes Fahren ist kein Spielraum für Fehler, es müssen noch einige Innovationen erprobt werden, um eine Gefährdung für anders Verkehrsteilnehmer zur Gänze ausschließen zu können. Die Bundesregierung hat daher das Digitale Testfeld Autobahn ins Leben gerufen, welches zukünftig noch um städtische und ländliche Testfelder erweitert werden soll. Hier sollen ausgiebige Testphasen umgesetzt werden, um mögliche Gefahren frühzeitig zu erkennen und zu eliminieren \cite{AutomatisiertesFahren}.