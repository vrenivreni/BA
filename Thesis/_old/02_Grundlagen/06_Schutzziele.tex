\section{Schutzziele einer Anwendung}\label{sec:Schutzziele}
In der Informationssicherheit gibt es das Konzept der Schutzziele. Dieses stellt sicher, dass sicherheitskritische Daten geschützt und unverändert übertragen werden. Der Zugriff auf die Daten ist zu beschränken und zu kontrollieren, sodass nur autorisierten Nutzern ein Zugriff gewährt wird. Die Schutzziele, die diese Anforderungen präzisieren, sind Integrität, Verfügbarkeit und Vertraulichkeit. Netzwerkgeräte, die auf kritische Daten zugreifen sollen, müssen je nach Sicherheitsanforderung eindeutig identifiziert werden können. Die entsprechende Eigenschaft der Geräte nennt man Authentizität, welches ein kombiniertes Schutzziel aus mehreren Parametern ist. Es ist eines der wichtigsten Ziele, da über die Authentifizierung der Zugriff auf Daten geregelt werden kann und somit die Verfügbarkeit gewährleistet wird \cite{Schutzziel}.

Die angesprochenen und für die Arbeit relevanten Schutzziele werden in den folgenden Unterabschnitten kurz erläutert.

\newpage

\subsection{Integrität}
Ein System gewährleistet die Datenintegrität, wenn es keinem Netzwerkteilnehmer möglich ist, die zu schützenden Daten unautorisiert zu verändern oder zu manipulieren.
Dies kann durch Festlegen von Rechten an Daten erfolgen. So kann eine Anwendung beispielsweise nur Leserechte an den übermittelten Daten haben, während eine andere sie auch bearbeiten darf.
Manipulation muss jederzeit nachgewiesen werden können und darf nicht unbemerkt bleiben. Dies kann z.B. anhand der bereits im Abschnitt \ref{sec:Ethernet} und \ref{sec:MacSec} erwähnten Check-Summen überprüft werden. Sind die Summen nicht korrekt, weist es auf eine Veränderung der Daten hin. Eine Veränderung kann sich einerseits durch einen Übertragungsfehler ergeben oder auch durch Manipulation der Daten.

\subsection{Verfügbarkeit}
Das Netzwerk gewährleistet eine gute Verfügbarkeit, wenn authentifizierte und autorisierte Netzwerkteilnehmer bei der Ausführung ihrer Aktionen nicht unautorisiert beeinträchtigt werden können. Die Erfüllung dieses Schutzziels benötigt jedoch eine Nutzungsverwaltung von Systemressourcen und ist nicht Teil dieser Arbeit.

\subsection{Vertraulichkeit}
Ein System weist eine Informationsvertraulichkeit auf, wenn es keine unautorisierten Informationsgewinnung ermöglicht. Dies wird mittels Kontrolle der Datenflüsse im Netzwerk gewährleistet. Eine Möglichkeit hierzu ist die Verwendung von einer Ende-zu-Ende Verschlüsselung. Auch dies erfordert eine eigene Technologie, welche nicht Teil dieser Arbeit ist.

\subsection{Authentizität}
Die Authentizität eines Objekt gibt dessen Echtheit und Glaubwürdigkeit an. Anhand dieser kann sicher gestellt werden, dass es sich um einen berechtigten Nutzer der Daten handelt. Dies wird durch eindeutige Identifikationsmerkmale, wie beispielsweise \acf{ID}, gewährleistet.