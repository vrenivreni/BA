\section{Ziel dieser Arbeit}
Die Erfinder dieser Lean-Prozesse werben damit, dass dadurch jedes Startup Erfolg haben würde. Es gibt auch bereits sehr viele große Unternehmen, die solche Strategien entweder für die Gründung oder für kleinere Erweiterungen benutzt haben. Dagegen stehen trotzdem die Scheiterungsstatistiken von Startups. Daher könnte es auch einfach Zufall sein und jene Unternehmen hätten sich auch ohne Lean-Methoden gut entwickelt. Es lässt sich leider keine Aussage darüber treffen, wie viele der Lean-Startups am Ende doch scheitern und warum. Doch zusammenfassend kann man sagen, dass der Trend eindeutig in die Richtung geht, kreative Prozesse zu benutzen. Der Mehrwert wird hoch geschätzt, allerdings kommen diese Aussagen hauptsächlich von den Prozessgründern.

Das Ziel dieser Arbeit ist es daher, zu erarbeiten, ob moderne Lean-Ansätze bei der Gründung eines Startups sinnvoll sind. 