\section{Motivation der Arbeit}
Es ist statistisch erwiesen, dass rund 70\% aller IT-Startups scheitern. \cite{CBInsights_failure} Diese Statistik steht dem Scheinbild entgegen, nach welchem jede Idee Gold wert ist. Die Idee alleine genügt allerdings nicht, denn es ist auch essentiell, diese auf die Zielgruppe abzustimmen und richtig zu vermarkten. Die Hauptgründe für das Scheitern der meisten Unternehmen sind sehr ähnlich. Eine der Ursachen dafür ist beispielsweise die fehlende Nachfrage. Außerdem war bei ungefähr 17\% aller gescheiterten Startups die schlechte Benutzerfreundlichkeit des Produktes ein Problem. Weitere Schwierigkeiten stellen etwa das falsche Team, fehlendes Budget, Konkurrenzkampf usw. dar. \cite{CBInsights_reasons}

Darüber hinaus ist es hauptsächlich in der IT-Branche ein verbreiteter Fehler, dass sich Entwickler sofort auf die Implementierung stürzen, ohne die Rahmenbedingungen zuerst genau abzustecken. Diese denken oft, sie haben ein klares Bild vom Endprodukt, ohne das wirklich mit dem Kunden (oder der potentiellen Zielgruppe) abgestimmt zu haben. Bei einem Startup gibt es zunächst noch keinen konkreten Kunden. Deshalb muss auf potentielle Zielgruppen zugegangen werden, was auch weitestgehend vermieden wird. Am Ende wird das Produkt aus Entwicklersicht umgesetzt, was logischerweise in fehlender Nachfrage resultieren kann, da es meist nicht dem entspricht, was wirklich gebraucht wird. Daraus kann sich eine Kette von Problemen entwickeln. Dass bei fehlendem Erfolg das Klima im Team zu leiden hat und auch die Sponsoren ausbleiben, ist ersichtlich. Entwickelt zeitgleich ein anderes Team ein ähnliches Produkt und schafft jenes eine bessere Umsetzung, wurde man klar von der Konkurrenz verdrängt.

Besonders für IT Interessenten und Programmierer werden an verschiedenen Standorten regelmäßige Hackathons veranstaltet. Dies ist ein Event, welches an einem Wochenende kreative Programmierer oder Interessierte an verschiedenen Projekten arbeiten lässt. Dazu werden teilweise von Firmen Challenges gestellt, welche zu lösen sind. Dazu gibt es die Möglichkeit, eine eigenen Idee umzusetzen und vorzustellen. Die besten Ideen bzw. Umsetzungen werden am Ende mit einem Preis belohnt. 

An einem Hackathon in Regensburg formt sich ein junges Team aus Studenten, welches eine eigenen Idee umsetzt. Schon bald fällt auf, dass die Umsetzung einer einfachen Idee komplizierter ist, als ursprünglich angenommen. Nichtsdestotrotz schafft es das Team an diesem Wochenende, einen ersten Prototypen im Rahmen einer Android-App zu erstellen. Damit kann die Gruppe die Jury in zwei Runden überzeugen und gewinnt so den ersten Preis des Hackaburgs. Aufgrund des positiven Feedbacks und dem hohen Engagement im Team entscheiden die Teilnehmer, die Idee weiter zu verfolgen. Daraus entsteht der Beschluss, ein Startup um die vermeintlich lukrative Idee zu gründen.

Um den oben beschriebenen typischen Startup-Schwierigkeiten entgegenzuwirken entstehen immer mehr moderne Prozesse, mit welchen jede Idee vermarktbar und erfolgreich zu sein scheint. Die Gründer versprechen sehr großen Erfolg und auch in immer mehr Unternehmen werden diese Innovation-Prozesse angewandt. Diese bauen alle auf einem gleichen Prinzip auf: Das Produkt perfekt auf den Kunden abzustimmen.

Es stellt sich also die Frage, ob potentiell vermeidbare Probleme bei der Startup-Gründung mithilfe von modernen und erfolgsversprechenden Prozessen eliminiert werden können. Um dies zu beantworten wird die Prozesskette ,,The Innovator's Method'', welche die berühmtesten modernen Methoden enthält, mit dem oben beschriebenen Startup-Team durchlaufen. Anhand dieses Beispiels sollen die Prozesse evaluiert werden und diese Ergebnisse am Ende mit anderen Erfahrungsberichten verglichen. 