\chapter*{Abstract}
Diese Arbeit befasst sich mit der Anwendung von Lean-Konzepten bei der Startup-Gründung. Die Hauptursache für das Scheitern von Unternehmensgründungen ist eine fehlende Nachfrage auf dem Markt. Der Fokus auf die Wünsche der Zielgruppe, welcher in den Lean-Ansätzen verankert ist, soll das Risiko fehlender Kunden im Vorhinein minimieren. Im Detail werden die Prozesse \textit{The Innovator's Method}, \textit{Sprint}, \textit{The Lean Startup} und die Erstellung einer \textit{Business Model Canvas} untersucht. Das IT-Startup \textit{Agrishare} durchläuft beispielgebend die genannten Prozesse und bewertet diese. Obwohl der Erfolg von Lean-Konzepten nicht mit Zahlen belegt werden kann, lässt die Umsetzung durch Agrishare darauf schließen, dass diese Ansätze durchaus sinnvoll sind. Der Hauptgrund für die positive Bewertung genannter Prozesse ist die iterative Implementierung des Produktes, wodurch bereits im Entwicklungsprozess grundlegende Schwachstellen elimiert werden können. Außerdem kann durch die in dieser Arbeit beschriebenen Methoden eine Idee innerhalb kurzer Zeit umgesetzt und ein Geschäftsmodell dazu entwickelt werden.