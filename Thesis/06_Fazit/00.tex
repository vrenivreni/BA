\chapter{Schlusswort}
\section{Zusammenfassung}
Nachdem sich Agrishare im Rahmen des Hackaburgs bildet, steht bereits ein grobes Konzept. Demnach soll eine Online-Plattform die Ressourcenauslastung in der Agrarbranche effizienter gestalten. Nach dem Erfolg bei dem Event beschließt das Team, ein Startup zu gründen, um die Plattform auf den Markt zu bringen. Nach gründlicher Recherche zur Gründung eines IT-Unternehmens im Allgemeinen, stößt die Gruppe auf eine Reihe erfolgsversprechender Konzepte. So auch die in Kapitel \ref{sec:Grundlagen} beschriebenen Prozesse. 

Aufbauend auf die Innovator's Method in Kapitel \ref{TheInnovatorsMethod} bekommt die Gruppe bereits wertvolle Einblicke in die Kundensicht. So wird klar, dass in der Agrarbranche ein Weg, schnell und effizient Ressourcen zu vermitteln, benötigt wird. Mit diesem Wissen startet das Team einen fünftägigen Workshop nach dem Prinzip des Sprint von Eric Ries, welcher in Kapitel \ref{sec:Sprint-Grundlagen} zusammengefasst erlärt ist. Die Umsetzung ist in Kapitel \ref{sec:Sprint-Umsetzung} genauer beschrieben. Allerdings wird der Prozess etwa ab der Hälfte dahingehend abgewandelt, dass die Entwicklung des Prototypen verlängert wird. Der Grund dafür ist, dass dies den ersten Prototypen auf Webbasis darstellt. Daher ist es wichtig, die einzelnen Bestandteile einheitlich aufzubauen, was nicht innerhalb der vorgegebenen Zeit realisierbar ist. Trotzdem schafft es das Team, durch den Sprint viele Details im Vorhinein abzuklären. So können die unterschiedlichen Ansprüche der Mitglieder an das Produkt, welche überraschenderweise sehr voneinander abweichen, miteinander vereinbart werden. Außerdem werden wichtige Begriffe eindeutig definiert, sodass diese zukünftig nicht für Verwirrungen im Team sorgen. Viele andere Punkte, welche unter Punkt \ref{sec:Sprint-Umsetzung} genauer erklärt sind, sorgen für ein durchgehend positives Feedback des Teams. Der Workshop wird generell als sehr hilfreich angesehen, obwohl er in diesem Fall abgeändert werden muss. 

Mit den Erkenntnissen aus dem Sprint und den anschließenden Tests können Konzepte der Lean-Startup-Methode angewandt werden. Der Prototyp aus dem Sprint-Prozess definiert die Grundlage, womit das Innovation Accounting Konzept in Kapitel \ref{sec:LeanStartup_InnovationAccounting} grundlegend beginnt. Allerdings fällt bereits in den ersten Nutzer-Tests auf, dass die tatsächliche Problematik ausschließlich in der Vermittlung der Ressourcen liegt. Die ursprüngliche Hypothese, nach welcher zusätzlich eine reibungslose Transaktion benötigt wird, wird von den Kunden widerlegt. An dieser Stelle wird zum ersten Mal das Prinzip des Validated Learning angewandt, welches in Kapitel \ref{sec:LeanStartup} erklärt wird. Daraus resultiert die Entscheidung ein Pivot durchzuführen. So wird die Plattform lediglich für die Vermittlung entwickelt, wie in Kapitel \ref{sec:LeanStartup_Pivot} eingehend beschrieben ist. Außerdem zeigt sich in einer weiteren Iteration der \ac{BML} Schleife, dass Vorschläge für attraktive Angebote auf der Startseite positiv auf die Endnutzer wirkt. Daraus entsteht mittels der in Kapitel \ref{sec:LeanStartup} beschriebenen Konzepte ein strukturiertes Konzept. 

Damit ist Schritt drei der in Kapitel \ref{sec:TheInnovatorsMethod} auf Abbildung \ref{fig:TheInnovatorsMethod} dargestellten Konzeptfolge abgeschlossen. So kann auf diesem Ergebnis aufbauend ein Geschäftsmodell entwickelt werden. Dieses wird nach dem in der Innovator's Method vorgeschlagenem Prinzip der Business Model Canvas abgebildet. Ein erster Entwurf dieser ist in Abbildung \ref{BMC_Structure} dargestellt. Diese wird nach dem in Kapitel \ref{BMC_Kapitel} beschriebenen Aufbau erstellt. Im weiteren Verlauf des Projektes wird die Canvas voraussichtlich weiter angepasst.

\section{Fazit und Ausblick}
Es ist unmöglich, den Erfolg der in dieser Arbeit vorgestellten Konzepte mit Zahlen zu belegen. Denn, scheitert ein Startup, kann das unterschiedliche Gründe haben. Das ist nicht zwingend abhängig davon, welche Konzepte bei der Gründung angewandt werden. Darüber hinaus ist der Erfolg eines Unternehmens auch nicht ausschließlich auf intern angewendete Prozesse zurückzuführen. So wird ein Produkt, wofür es keinerlei Nachfrage auf dem Markt gibt, unabhängig von Konzepten oder Marketingstrategien scheitern. Daher ist ein Grundgedanke der Lean-Methoden, die Nachfrage durch die Kundenorientierung im Vorhinein zu prüfen. Damit soll vorab verhindert werden, dass ein Unternehmen gegründet wird, wessen Produkt nicht benötigt wird. Wird ein Prototyp in den Usertests durchgehend schlecht bewertet, fällt idealerweise bereits zu Beginn des Startups auf, dass das Produkt nicht angenommen wird und somit eine Unternehmensgründung wahrscheinlich fehlschlagen wird. Im Umkehrschluss sollten dabei die Scheiterungsstatistiken von Startups zurückgehen, was nicht der Fall ist. Dabei stellt sich die Frage, wie viele von den gescheiterten Startups Lean Methoden angewandt haben. Auch dafür können keine konkreten Zahlen angegeben werden. Ein junges Unternehmen, welches trotz der Anwendung dieser Methoden scheitert, könnte die Konzepte unter Umständen auch unvollständig und inkonsequent angewandt haben. Dann würden diese fälschlicherweise die Erfolsrate der Lean-Konzepte verschlechtern. 

Was hingegen im kritischen Umgang mit modernen Design-Prozessen auffällt, ist, dass der größte Nachteil derer die aufzuwendende Zeit ist. Denn das Team stimmt zuvor das Design des Produktes mit der Zielgruppe ab, anstatt sofort mit der Implementierung zu beginnen. Bekommt ein Team also im Rahmen der einzelnen Iterationen keine neuen Einblicke in die Bedürfnisse der Kunden, können diese als unnötig angesehen werden. Jedoch scheint es utopisch, ein von Anfang an perfekt auf den Kunden abgestimmtes Produktkonzept zu haben, welches sich auch in Usertests als ideal erweist. Trotzdem ist es nicht unwahrscheinlich, dass sich eine gute Idee auch ohne bekannte Lean-Prozesse vermarkten lässt. Das Agrishare Team, sowie zahlreiche weitere Beispiele der Autoren oben beschriebener Konzepte zeigen exemplarisch, dass die Abstimmung auf die Zeilgruppe in den meisten Fällen sinnvoll ist. So hätte Agrishare vermutlich viel Zeit in die Umsetzung der Transaktion investiert, welche sich als unnötig herausgestellt hat. Durch die Lean Startup Methoden konnte dies zuvor eliminiert werden. So wird im Ergebnis Implementierungszeit und -aufwand gespart durch die Anwendung des Lean Konzeptes. Ähnliche Ergebnisse zeigen sich auch an den Beispielen der Sprint- und Lean Startup-Methoden. Daher lässt sich grundlegend feststellen, dass die Design-Prozesse durchaus sinnvoll sind.

Im weiteren Verlauf der Startup-Gründung am Beispiel Agrishare beschließt das Team, die getesteten Konzepte weiterhin umzusetzen und in die weitere Unternehmensstruktur einzubauen. Da die Plattform derzeit noch nicht auf dem Markt ist, lassen sich noch keine Aussagen über Wachstum oder Erfolg machen. Das Team ist allerdings sehr zuversichtlich, mithilfe der Kundennähe langfristigen Erfolg mit Agrishare zu haben.