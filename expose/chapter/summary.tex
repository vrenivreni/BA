%\textbf{Hinweise zum Titel der Abschlussarbeit:}
%Der Titel der Abschlussarbeit ist deren „Aushängeschild“ und daher sehr wichtig. Er soll prägnant und verständlich formuliert sein. Folgende Fragen sollten helfen dies zu erreichen:
%\begin{itemize}
%\item Trifft der Titel den geplanten Inhalt der Arbeit? Ist der Titel kurz (maximal Zweizeiler), prägnant und allgemein verständlich?
%\item Wird nur eine Sprache verwendet (deutsch oder englisch und nicht denglisch)? Sind die verwendeten Abkürzungen allgemein bekannt oder können Sie nicht auch vermieden werden?
%\item Würden Sie sich selbst für die Arbeit nur aufgrund des gewählten Titels interessieren und diese lesen wollen?
%\item Würde ihr zukünftiger Arbeitgeber den Titel verstehen?
%\end{itemize}
%\textbf{Hinweise zum Inhalt des Exposè:}
%
%Das Exposé stellt die Grundlage für ein Arbeitsvorhaben dar, ist die Voraussetzung für die
%Anmeldung zur Abschlussarbeit. Das Exposé dient sowohl der eigenen Orientierung als auch der Verständigung zwischen
%Kandidatin / Kandidaten und Prüferin/ Prüfer. Auf ein bis zwei Seiten sollten zuerst folgende Punkte erläutert werden (Nicht alle Punkte
%sind bei jedem Thema relevant). Die nicht selbstständig getroffenen Aussagen sind mit Literaturquellen zu belegen.
\section{Problemstellung}
%\begin{itemize}
%\item Welches wissenschaftlich oder fachlich relevante Problem ist der Ausgangspunkt der Arbeit und warum handelt es sich dabei um ein Problem? 
%\item Hohe Quote für das Scheitern von IT Startups aufgrund von vermeidbaren Dingen.
%\item Welche Relevanz hat das untersuchte Problem?
%\item Warum ist es lohnenswert diesem Problem nachzugehen? Warum soll ausgerechnet dieses Problem behandelt werden?
%\item 
%\item Wie kommt es zu dem Thema?
%\end{itemize}
%In der heutigen Welt hat jeder die Möglichkeit, sich selbstständig zu machen. Meist reicht ein kleines Team und eine gute Idee aus. Viele heute sehr große Unternehmen sind aus einem Startup entstanden und unglaublich erfolgreich geworden, wie zum Beispiel Facebook oder AirBnb. Dadurch ist für viele das Scheinbild entstanden, dass jede noch so einfache Idee zu einem Milliarden-Unternehmen werden könnte. 
Es ist statistisch erwiesen, dass rund 70\% aller IT-Startups scheitern. \citep{CBInsights_failure} Diese Statistik steht dem Scheinbild entgegen, nach welchem jede Idee Gold wert ist. Die Idee alleine genügt leider nicht, denn es ist auch essentiell, diese auf die Zielgruppe abzustimmen und richtig zu vermarkten. Die Gründe für das Scheitern der meisten Unternehmen sind oft sehr ähnlich. Einer der Hauptgründe dafür ist beispielsweise die fehlende Nachfrage. Außerdem war bei ungefähr 17\% aller gescheiterten Startups die schlechte Benutzerfreundlichkeit des Produktes ein Problem. Weitere Probleme stellen beispielsweise das falsche Team, fehlendes Budget, Konkurrenzkampf usw. dar. \citep{CBInsights_reasons}

Darüber hinaus ist es hauptsächlich in der IT-Branche ist es ein verbreiteter Fehler, dass sich Entwickler sofort auf die Implementierung stürzen, ohne die Rahmenbedingungen zuerst genau abzustecken. Diese denken oft, sie haben ein klares Bild vom Endprodukt, ohne das wirklich mit dem Kunden (oder der potentiellen Zielgruppe) abgestimmt zu haben. Bei einem Startup gibt es zunächst noch keinen konkreten Kunden. Deshalb muss auf potentielle Zielgruppen zugegangen werden, was auch weitestgehend vermieden wird. Am Ende wird das Produkt aus Entwicklersicht umgesetzt, was logischerweise in fehlender Nachfrage resultieren kann, da es meist nicht dem entspricht, was wirklich gebraucht wird. Daraus kann sich nun eine Kette von Problemen entwickeln. Dass bei fehlendem Erfolg das Klima im Team zu leiden hat und auch die Sponsoren ausbleiben, ist ersichtlich. Entwickelt zeitgleich ein anderes Team ein ähnliches Produkt und schafft jenes eine bessere Umsetzung, wurde man klar von der Konkurrenz verdrängt.

\section{Fragestellung}
%\begin{itemize}
%\item Was genau werden Sie selbst untersuchen?
%\item Mit diesem Schritt soll das Thema weiter eingegrenzt werden.
%\item Auf welche zentrale Frage soll in der Arbeit eine Antwort gefunden oder gegeben werden?
%\item Welches konkrete Problem soll damit (aus welcher Perspektive und unter welchen Vorzeichen) behandelt werden? 
%\item Hier sollte eine Problemanalyse durchgeführt werden und Teilprobleme identifiziert werden.
%\end{itemize}
Um den oben beschriebenen Problemen entgegenzuwirken entstehen immer mehr moderne Prozesse, mit welchen jede Idee vermarktbar und erfolgreich zu sein scheint. Die Gründer versprechen sehr großen Erfolg und auch in immer mehr Unternehmen werden diese Innovation-Prozesse angewandt. Diese bauen alle auf einem gleichen Prinzip auf: Das Produkt perfekt auf den Kunden abzustimmen.

Es stellt sich also die Frage, ob potentiell vermeidbare Probleme bei der Startup-Gründung mithilfe von modernen und erfolgsversprechenden Prozessen eliminiert werden können. 

%\section{Ziele/ Hypothesen}
%\begin{itemize}
%\item Was soll mit den Ausführungen erreicht werden?
%\item Was soll belegt oder widerlegt werden?
%\item Beide Aspekte müssen mit den vorher aufgestellten (Leit-)Fragen übereinstimmen.
%\end{itemize}
Ziel dieser Arbeit soll es sein, eine Aussage darüber zu treffen, wie sinnvoll die Anwendung solcher Prozesse wirklich ist. Diese gelten dann als sinnvoll, wenn am Ende ein klarer Mehrwert erkennbar ist. Dieser Mehrwert soll anhand folgender Fragen gemessen werden: 
\begin{itemize}
	\itemsep0em
	\item Weicht der durch einen kreativen Workshop entstandene Prototyp sehr von der Anfangsidee ab?	
	\item Ist das Endprodukt besser als die initiale Idee der Entwickler?
	\item Können jene Prozesse auch anderen Startup-Problemen entgegenwirken?
\end{itemize}

\section{Theoriebezug / Forschungsstand}
%\begin{itemize}
%\item Welche wissenschaftlichen Erkenntnisse liegen zu dem Thema bereits vor?
%\item Welche Aspekte des Themas sind bisher noch nicht ausreichend oder erfolgreich behandelt worden? 
%\item Auf welche Begriffe, Theorien, Modelle oder Erklärungsansätze soll Bezug genommen werden?
%\end{itemize}
Die Erfinder dieser Prozesse werben damit, dass dadurch jedes Startup Erfolg haben würde. Es gibt auch bereits sehr viele große Unternehmen, die solche Strategien entweder für die Gründung oder für kleinere Erweiterungen benutzt haben. Dagegen stehen trotzdem die Scheiterungsstatistiken von Startups. Daher könnte es auch einfach Zufall sein und jene Unternehmen hätten sich auch ohne Lean-Methoden gut entwickelt. Es lässt sich leider keine Aussage darüber machen, wie viele der Lean-Startups am Ende doch scheitern und warum. Doch zusammenfassend kann man sagen, dass der Trend eindeutig in die Richtung geht, kreative Prozesse zu benutzen. Der Mehrwert wird hoch geschätzt, allerdings kommen diese Aussagen hauptsächlich von den Prozessgründern.

\section{Methode}
%\begin{itemize}
%\item Mit welchen wissenschaftlichen Methoden soll das Problem bzw. Teilprobleme bearbeitet werden?
%\item Welche Methoden bieten sich an, die (Leit-)Fragen und Hypothesen angemessen zu
%bearbeiten? (theoretisch oder empirisch, qualitativ oder quantitativ, eine Kombination der
%Methoden, etc.).
%\end{itemize}
Die oben beschriebenen Fragen sollen anhand des Prozesses SPRINT von Google Ventures gemessen werden. Ein junges Startup-Team wird diesen Sprint für die Entwicklung ihrer Leitidee durchlaufen. 

\section{Evaluierungsstrategie}
%\begin{itemize}
%\item Wie sollen die entwickelten Methoden evaluiert werden, so dass nachgewiesen werden
%kann, dass das / die Ziel(e) auch erreicht wurde(n).
%\end{itemize}
Der Mehrwert eines modernen Lean-basierten Prozesses kann in diesem Fall natürlich nicht am Erfolg eines neu gegründeten Startups gemessen werden, da dieser auf mehrere Jahre hinweg gemessen wird. Es soll ein Vergleich zwischen den Prototypen mit versus ohne Sprint stattfinden. Diese Ergebnisse werden am Ende mit potentiellen Kunden mittels eines Usability-Tests abgeglichen. Anhand der Auswertung dieser Tests wird der Erfolg des Sprints gemessen. Auch das Team selbst wird in die Evaluierung miteingebunden indem die Mitglieder abstimmen, ob sie den Prozess als effektiv oder überflüssig empfinden.
Weichen initiale Produktidee und Prototyp nach dem Sprint sehr weit voneinander ab, ist zu erwarten, dass eine klare Aussage getroffen wird, welches Produkt erfolgsversprechender ist. Sind sie sich allerdings sehr ähnlich, ist der Erfolg von SPRINT an diesem Beispiel trotzdem nicht klar widerlegt. In diesem Fall wird vom Team eine Aussage gemacht, ob der Prozess an sich sinnvoll ist. 





