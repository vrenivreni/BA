\section{Ablaufbeschreibung}
5 Tage

\subsection*{Tag 1}
\paragraph{Long Term Goal}
Der Sprint beginnt mit der Festlegung eines Langzeitzieles. Darunter versteht man einen Satz, der aussagt, was mit dem Projekt auf lange Sicht erreicht werden soll. In einer Gruppendiskussion sollen hier Fragen wie \textit{Warum wird das Projekt durchgeführt? Wo sehen wir uns in 6 Monaten, in einem Jahr oder sogar in fünf Jahren?} geklärt werden. Oft kann es hier dazu kommen, dass die Teammitglieder unterschiedliche Erwartungen haben. Diese Aufgabe ist also einerseits dafür gedacht, die Erwartungen des Gesamtteams zu einer realistischen Gesamterwartung zusammenzufassen. Andererseits soll dieses Langzeitziel auch während des gesamten Sprints dazu dienen, den Fokus nicht zu verlieren. Da man sich diesen Satz immer wieder ins Gedächtnis ruft, ist es einfacher, fokussiert zu bleiben. Damit dieser Satz den kompletten Sprint über sichtbar ist, wird er gut sichtbar auf ein Whiteboard notiert. 

\paragraph{Sprint-Fragen}
Nachdem die Erwatungen aller Teilnehmer zusammengefasst und geklärt wurden, werden im nächsten Schritt alle Bedenken gesammelt. Dafür werden offene Fragen aus der Runde auf ein weiteres Whiteboard geschrieben. Jeder Teilnehmer hat nun die Aufgabe, seine Bedenken oder Ängste zu einer Frage zu formulieren oder auch Fragen, die er gerne im Laufe der Zeit beantwortet haben möchte, zu stellen. Hierbei wird nicht aussortiert oder geurteilt, alle Beiträge aus dem Team werden notiert und somit auch über den Sprint-Zeitraum im Gedächtnis behalten. 

\paragraph{Map}
Die nächste Aufgabe besteht darin, das Gesamtprojekt auf einer Karte oder einem Graphen festzuhalten. Dieser Graph soll das Projekt möglichst einfach aber trotzdem verständlich darstellen. Festgehalten wird diese Karte auf dem ersten Whiteboard direkt unter dem Long Term Goal. Dabei wird mit den Schlüsselpersonen begonnen, welche auf der linken Seite untereinander gelistet werden. Darunter versteht man meist unterschiedliche Kundengruppen oder auch andere wichtige Personengruppen, wie beispielsweise der Staat. Danach wird rechts das Ende des Prozesses aufgeschrieben. Das soll aus einem Stichwort bestehen, welches das in gewisser Weise auch den Sinn des ganzen Projekts beschreibt, wie zum Beispiel \textit{Behandlung} oder \textit{Kauf}. Nachdem Start und Ziel des Prozesses festgehalten sind, sollen nun die Zwischenschritte eingetragen werden. Das beinhaltet jeden Schritt, den eine Schlüsselperson durchlaufen muss, um an das Ziel rechts zu kommen. Sehr wichtig ist es, diese Schritte zwar sehr abstrakt zu halten, aber trotzdem die wichtigsten Schritte zu erkennen und aufzuschreiben. Als Richtwert sollten am Ende zwischen fünf und 15 Schritte dargestellt sein.

Nach dieser Aufgabe sollte es bereits ca. 11:00 Uhr sein. Daher wird empfohlen, eine einstündige Mittagspause einzulegen.

\paragraph{Ask the Experts}
Für diese Aufgabe sollte man im Vorfeld Personen einladen, welche sich in verschiedenen Teilbereichen des Gesamtprojekts besonders gut auskennen. Diese können auch aus dem Sprint-Team selbst sein oder extern für diesen Nachmittag dazustoßen. Das hat den Vorteil, dass sich selbst die Projektleiter nicht in jedem Teilgebiet gut genug auskennen, um detaillierte Beschreibungen und Erklärungen zu liefern. Welche Experten man einladen sollte ist natürlich immer abhängig von dem Projekt. Allerdings gibt es grobe Vorschläge von Google Ventures.
Die Gebiete, aus denen man idealerweise einen Fachmann zu Rate ziehen sollte, sind folgende: 
\begin{itemize}
	\item \textbf{Strategie:}
	Es wird hier durchaus empfohlen, jemanden, bespielsweise den Decider, über die Gesamtstrategie vortragen zu lassen. Das dient dem Allgemeinverständnis des Teams über das Ziel und die Schritte zum Erreichen dieses Ziels. 
	\item \textbf{Kundensicht:}
	Diese Person sollte ein Mitarbeiter sein, welcher am meisten von der Kundenseite versteht. Dieser sollte auch in der Lage sein, die Sichtweise der Kunden präzise zu erklären und mögliche Risiken oder interessante Einsichten aufbringen.
	\item \textbf{Produktexperte:}
	Es ist außerdem wichtig, jemanden einzuladen, der die Produktseite vertritt, im Gegensatz zur Kundenseite. Dieser sollte die Mechanik des Endproduktes kennen und über Einzelheiten in der Produktion Bescheid wissen. Da es in diesem Feld sehr viele Einzelbereiche gibt, wie zum Beispiel Finanzen, Tech/Logistik oder Marketing, ist es üblich, mehrere Experten aus den einzelnen Bereichen einzuladen. Hier geht es vor allem darum, herauszufinden, wie diese Teile zusammenpassen können. Die Art und Anzahl der Experten hängt wiederum vom Projekt ab und kann deshalb nicht pauschal eingegrenzt werden.
	\item \textbf{Probleme in der Vergangenheit:}
	Eventuell gab es in der Vergangenheit bereits Personen oder Teams, welche sich intensiver mit dem Thema befasst haben. Dann könnten diese darüber berichten, welche Probleme aufgekommen sind oder ob es bereits Lösungsansätze gibt.
\end{itemize}

Um den zeitlichen Rahmen des Workshops einzuhalten, sollte man diese Expertenrunden auf 30 Minuten pro Person eingrenzen. Falls der Fachmann nicht Teil des Sprint-Teams ist, sollte zuerst der Grund für den Sprint erläutert werden und die Whiteboards mit dem Sprint-Ziel, der Map und den Fragen kurz nähergebracht werden. Danach sollte der Experte frei über sein Spezialgebiet im Bezug auf das Projekt erzählen. Das Team sollte dabei viele Fragen stellen, sodass jedes Mitglied ein möglichst tiefes Verständnis für diesen Bereich bekommt. Falls nötig, werden nun die Whiteboards abgeändert. Das heißt, falls sich Änderung auftun, welche nicht mit dem Ziel oder der Map vereinbar sind, dürfen diese an der Stelle angepasst werden. Außerdem sollten wichtige Fragen bei den Sprint-Fragen hinzugefügt werden.

Während der Expertenrunde soll das gesamte Team nicht nur Fragen stellen, sondern auch Notizen machen, jede Person separat. Dafür wird eine Vorgehensweise vorgeschlagen, die \textit{How Might We} genannt wird. Dabei bekommt jede Person einen Block mit Haftnotizen und einen schwarzen Marker. In die linke obere Ecke werden die Buchstaben HMW geschrieben, um den Fragesatz einzuleiten. Wenn man etwas interessantes hört, wird diese Information als Frage formuliert auf die Haftnotiz geschrieben und diese Notiz beiseite gelegt.

\paragraph{Notizen sortieren}
Nachdem die Expertenrunden abgeschlossen sind, werden die Haftnotizen von allen Teammitgliedern ungeordnet an eine freie Wand geklebt. Dann ordnet das Team die Notizen zu logischen Gruppen und findet eine passende Überschrift für die Kategorie. Typischerweise können die restlichen Notizen ohne logischen Zusammenhang zu der Gruppe \textit{Sonstige} zusammengefasst werden.

\paragraph{Notizen bewerten}
Die geordneten Notizen sollen nun priorisiert werden. Dafür bekommt jedes Teammitglied zwei Sticker und der Decider vier. Nachdem sich alle das Gesamtziel und die Fragen erneut vor Augen gerufen haben, sollen sie die Sticker still auf jene Notizen kleben, welche sie für am wichtigsten erachten. Dabei darf die eigene Notiz gewählt werden und es dürfen auf eine Notiz auch mehrere Sticker geklebt werden. Nach der stillen Abstimmung werden die Notizen mit den meisten Stickern zu den logisch passenden Schritten auf der Map geklebt.

\paragraph{Fokus des Sprints}
Zum Abschluss des ersten Sprint-Tages wird der Fokus für die restlichen Tage festgelegt. Dieser Fokus wird alleine vom Decider beschlossen, allerdings darf er das Team um Hilfe bitten. Dafür schreibt jedes Teammitglieder die Schritte der Map auf, die für ihn am wichtigsten sind. Die Vorschläge werden daraufhin auf einem Whiteboard gesammelt und kurz diskutiert, falls es stark abweichende Meinungen gibt. Da dies genug Input für den Decider sein soll, muss dieser nun auf der Map eine Zielgruppe und einen Zielschritt einkreisen. Danach werden wiederum die Sprint-Fragen wiederholt und jede markiert, die nach der Fokusauswahl im Rahmen des Sprints als lösbar scheinen.

\subsection*{Tag 2}
Der Vormittag des zweiten Tages läuft unter dem Motto \textit{Remix and Improve}. Es geht darum, bewährte Methoden oder Teile anderer Produkte auf das Projekt zuzuschneiden oder anzupassen. Wichtig dabei ist es, auch innerhalb anderer Branchen zu suchen und kreativ zu sein. Meist ist es nicht auf den ersten Blick erkennbar, wie andere Produkte auf das eigene Projekt angepasst werden sollen. Oft ist es aber nicht nötig, das Rad neu zu erfinden, da man bereits existierende Einzelteile einfach passend einbauen muss.

\paragraph{Lighting Demos}
Zuerst wird eine Liste mit Produkten erstellt, die Parallelen zu dem zu entwickelnden Produkt enthalten. Jedes Teammitglied soll sich dazu Gedanken machen und Vorschläge liefern. Die Liste soll außerdem auch Vorschläge aus anderen Branchen enthalten. Pro Person sollte die Liste ein bis zwei Produkte enthalten, welche dann auch von den jeweiligen Teammitgliedern vorgetragen werden.

Nachdem die Liste fertiggestellt wurde, stellt jeder Teilnehmer sein(e) Produkt(e) innerhalb von drei Minuten vor. Dabei soll die Person hauptsächlich auf für das Projekt relevante Bestandteile eingehen. Damit die Demo für alle gut sichtbar ist, ist es ratsam, den Screen eines Laptops an eine Wand zu projizieren und die Produkte dort zu zeigen. Der Facilitator zeichnet währenddessen die besonders guten Ideen an ein weiteres Whiteboard, gibt der Skizze eine Überschrift und notiert die Quelle darunter. So ist es für das gesamte Team einfacher, sich einerseits auf die Demos zu konzentrieren, andererseits kann man sich darauf verlassen, dass die Ideen festgehalten und so nicht vergessen werden.

\paragraph{Aufgabenverteilung}
Diese Aufgabe bedeutet, die einzelnen Bereiche des Prototypen aufzuteilen. Falls der Prototyp nur aus einem einzigen Teil besteht, kann auch das gesamte Team an der gleichen Sache arbeiten. Wenn das aber nicht der Fall ist, suchst sich zunächst jeder Teilnehmer aus, woran er gerne arbeiten würde. Ist die Aufteilung ungerecht, finden sich optimalerweise freiwillige Wechsler.

Danach sollte das Team nun wieder eine einstündige Pause einlegen um neue Energie für den Nachmittag zu tanken.

\paragraph{Sketch}
Sketching bedeutet, eine Idee auf einem großen Bogen Papier aufzuzeichnen. Das heißt, der gesamte Nachmittag des zweiten Tages besteht daraus, für sich selber einen Prototypen zu zeichnen. Die Skizze soll möglichst ohne schriftliche Erklärungen auskommen und gut verständlich sein, aber es muss sich um kein künstlerisches Meisterwerk handeln. Außerdem ist es einfacher, abstrakte Ideen durch Skizzen zu erklären als durch Worte. Diese Aufgabe wird wiederum alleine durchgeführt. Dadurch hat jedes Teammitglied die Chance, sich selbst Inspiration zu holen und in Tiefe über das Problem und eine geeignete Lösung nachzudenken. Damit jede Person konzentriert arbeitet und nicht so leicht abgelenkt wird, ist die Aufgabe in kleinere Unteraufgaben aufgeteilt. In den folgenden vier Schritten soll nun jede Person einen eigenen Sketch entwickeln:
\begin{enumerate}
	\item \textbf{Notes:} 20 Minuten
	
	In Schritt 1 soll jeder Teilnehmer durch den Sprint-Raum gehen und zuerst das Ziel auf ein Stück Papier schreiben. Dann soll dieser weitere Notizen sammeln indem er sich die Sprint-Fragen, die Map und alle gesammelten Notizen erneut vor Augen führt. Außerdem ist es hier auch erlaubt, Smartphones oder Laptops zu benutzen, um sich Inspiration oder zusätzliche Informationen zu holen. Innerhalb der letzten drei Minuten werden die wichtigsten Notizen dann markiert.
	\item \textbf{Ideas:} 20 Minuten
	
	Dieser Schritt dient dazu, Ideen zu entwickeln. Es sollen möglichst viele unterschiedliche Skizzen entstehen. Dazu zählen zum Beispiel kleine Zeichnungen, Beispielüberschriften, Diagramme, Strichfiguren oder Ähnliche. Es geht hauptsächlich darum, kreativ zu sein und viele unterschiedliche Ansätze zu schaffen. Am Ende werden wiederum die besten Skizzen markiert.
	\item \textbf{Crazy 8s:} 8 Minuten
	
	Diese Aufgabe 	besteht daraus, innerhalb von 8 Minuten die beste Idee aus dem Schritt vorher in 8 unterschiedliche Variationen zu skizzieren. Hierbei bleibt nicht viel Zeit für große Überlegungen, daher entstehen viele spontane Ideen.
	\item \textbf{Lösungs-Sketch:} 30+ Minuten
	
	Im letzten Schritt erstellt jedes Teammitglied den finalen Sketch. Jede Person sucht zunächst die beste Idee aus den vorherigen Schritten aus und versucht, diese auszuarbeiten. Diese Skizzen sollen das Format eines dreistufigen Storyboards haben, denn ein Produkt besteht nie aus nur einem Bild. Stattdessen interagiert der Kunde mit dem Produkt, was immer mehrere Stufen enthält. Falls der Sprint-Fokus so eingeschränkt ist, dass es sich tatsächlich nur um einen kleinen Teil des Gesamtprojektes handelt und es daher sinnvoller ist, sich auf eine Seite zu beschränken, kann von dem Storyboard natürlich abgewichen werden. Auf jeden Fall sollten die Sketche aber selbsterklärend und anonym sein. Dabei darf es sich um ganz einfache Zeichnungen handeln, wobei die Wortwahl allerdings sehr wichtig ist. Am Ende soll jede Zeichnung einen eindringlichen Titel haben.
\end{enumerate}

\subsection*{Tag 3}
Am dritten Tag wird eine Entscheidung darüber getroffen, welcher Lösungssketch am folgenden Tag prototypisiert wird. Dafür gibt es viele einzelne Schritte, umd zu vermeiden, dass endlose Diskussionen hervorgerufen werden oder die Person mit dem meisten Überzeugungstalent seinen Sketch am besten verkauft.

\paragraph{Kunstmuseum}
Zuerst werden alle Sketche nebeneinander an einer Wand angebracht. Dabei soll zwischen den Skizzan noch etwas Platz frei sein, ähnlich wie in einem Kunstmuseum. Falls möglich, können die Sketche auch in chronologischer Reihenfolge angebracht werden.

\paragraph{Heat map}
Nun wird darauf verzichtet, jeden Sketch einzeln zu erklären. Da die Zeichnungen ohnehin selbsterklärend sein sollen, das Überzeugungsgeschick des Zeichners außer Acht gelassen werden soll und Diskussionen möglichst vermieden werden sollen, wird sich jede Person zunächst still selbst einen Eindruck aller Sketche verschaffen. Dafür werden 20 bis 30 kleine Sticker an alle Teilnehmer verteilt. Diese werden an besonders gute Teile der Sketche geklebt und falls eine Idee besonders heraussticht, können dort auch 2-3 Sticker angebracht werden. Falls ein Sketch Fragen aufbringt, werden diese auf eine Haftnotiz unter die Zeichnung geklebt. Der Name der Übung sollte nun klar sein, da die besonders guten Details durch viele Sticker hevorgehoben werden. 

\paragraph{Speed Critique}
Diese Übung folgt wiederum einer klaren Struktur und Timeboxing wird durchaus empfohlen. Das Zeitlimit beträgt drei Minuten pro Sketch. Der Facilitator trägt einen Sketch vor und betont besonders herausstechende Teile. Auch die Teammitglieder dürfen besonders wichtige Details aufbringen, welche der Facilitator eventuell vergessen hat. Ein Freiwilliger aus dem Team sollte während dieser Übung alle wichtigen Dinge auf Haftnotizen festhalten und über dem Sketch anbringen. Während der restlichen Zeit werden noch Fragen und Bedenken im Team geklärt. Bis zu diesem Zeitpunkt bleibt der Zeichner dieses Sketches still und betont erst am Ende noch nicht aufgebrachte Details und beantwortet Fragen. Danach wird derselbe Prozess am nächsten Sketch angewandt.

\paragraph{Straw Poll}
Nun bekommt jeder Teilnehmer einen großen Sticker. Das Team wird dazu aufgerufen, das Ziel und die Sprint-Fragen nochmal für sich zu wiederholen. Innerhalb der nächsten zehn Minuten schreibt jede Person seine Entscheidung auf, welche ein gesamter Sketch oder auch nur ein kleiner Teil eines Sketches sein kann. Nach den zehn Minuten klebt jedes Teammitglied seinen Sticker auf einen der Sketche und erklärt seine Stimme kurz.

\paragraph{Supervote}
Die endgültige Entscheidung wird allerdings von den Decidern getroffen. Dafür bekommt jeder Decider drei große Sticker mit deren Initialen. Diese Sticker sollen nun an den Sketches angebracht werden, welche am nächsten Tag in einen Prototyp verwandelt werden sollen. Hier ist wichtig, dass dabei nur die Decider-Stimmen zählen, die Stimmen des restlichen Teams sollen die Decider nur in ihrer Entscheidung unterstützen. Alle Sketches mit Supervotes werden nun nebeneinander an einer Wand angebracht, alle anderen bleiben trotzdem an der Wand.

\subsection*{Tag 4}

\subsection*{Tag 5}

\subsection*{Resultat}
