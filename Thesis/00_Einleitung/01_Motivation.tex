\section{Motivation der Arbeit}
Es ist statistisch erwiesen, dass rund 70\% aller IT-Startups scheitern. \cite{CBInsights_failure} Diese Statistik steht dem Scheinbild entgegen, dass aus jeder simplen Idee ein Unternehmen entstehen kann. Doch die Idee alleine genügt nicht. Es ist auch wichtig, diese auf die Zielgruppe abzustimmen und richtig zu vermarkten. Die Hauptgründe für das Scheitern der meisten Unternehmen sind sehr ähnlich. Eine der Ursachen dafür ist beispielsweise die fehlende Nachfrage. Außerdem ist bei ungefähr 17\% aller gescheiterten Startups die schlechte Benutzerfreundlichkeit des Produktes ein Problem. Weitere Schwierigkeiten stellen etwa das falsche Team, fehlendes Budget, Konkurrenzkampf usw. dar. \cite{CBInsights_reasons}

Darüber hinaus ist es ein verbreiteter Fehler, dass bereits mit der Implementierung begonnen wird, bevor die Rahmenbedingungen genau festgelegt wurden. Die Entwickler denken oft, ein klares Bild vom Endprodukt zu haben, ohne es mit dem Kunden oder der potentiellen Zielgruppe abzustimmen. Daher wird das Produkt aus Entwicklersicht umgesetzt, was logischerweise in fehlender Nachfrage resultieren kann, da es meist nicht dem entspricht, was wirklich gebraucht wird.

Um den oben beschriebenen typischen Startup-Schwierigkeiten entgegenzuwirken entstehen neue Prozesse, mit welchen jede Idee vermarktbar und erfolgreich zu sein scheint. Die Erfinder dieser Konzepte versprechen sehr großen Erfolg und in immer mehr Unternehmen werden die Innovationsprozesse angewandt. Diese bauen alle auf einem gleichen Prinzip auf: Das Produkt perfekt auf den Kunden abzustimmen indem die Nutzer bereits in die Entwicklung miteingebunden werden. Dass ein Startup nur erfolgreich sein kann, wenn ein ausreichend großer Markt für das Produkt existiert, ist ersichtlich. So wird dies auch als der wichtigste Aspekt erfolgreicher Unternehmensgründungen genannt. Daher ist anzunehmen, dass Lean Prozesse die Anzahl der scheiternden Startups reduziert. Das entspricht allerdings nicht der Realität, weshalb diese Konzepte im Zuge dieser Arbeit kritisch betrachtet werden.\cite{patel201590} Diese werden unter Anderem von einem Beispielstartup getestet und darauf aufbauend Rückschlüsse gezogen.

Ein neues Startup-Team namens Agrishare erklärt sich bereit, einige der Prozesse zu durchlaufen und zu testen. Im Rahmen dieser Arbeit wird unter Anderem die Umsetzung der Konzepte anhand dieses Teams beschrieben. Außerdem fließen die Meinungen und Ergebnisse von Agrishare in die Bewertung der Prozesse mit ein.

Das Team besteht aus Studenten der OTH Regensburg und formt sich während eines Hackathons in Regensburg. Das ist ein Event, an welchem über einen festen Zeitraum verschiedene Software- oder Hardwareprobleme gelöst werden. Die Projekte werden teilweise von Firmen gestellt. Zusätzlich ist es möglich, eine eigene Idee umzusetzen und vorzustellen. Die besten Ausarbeitungen werden am Ende mit einem Preis belohnt, wie die Android-App des oben genannten Teams. Agrishre kann die Jury in zwei Runden überzeugen und gewinnt so den ersten Preis des Hackaburgs - wie die Veranstaltung in Regensburg offiziell heißt. Aufgrund des positiven Feedbacks und dem hohen Engagement im Team entscheiden die Teilnehmer, die Idee weiter zu verfolgen. Daraus entsteht der Beschluss, ein Startup um das vielversprechende Projekt zu gründen.

%Es stellt sich also die Frage, ob potentiell vermeidbare Probleme bei der Startup-Gründung mithilfe von modernen und erfolgsversprechenden Prozessen eliminiert werden können. Um dies zu beantworten wird die Prozesskette ,,The Innovator's Method'', welche die berühmtesten modernen Methoden enthält, mit dem oben beschriebenen Startup-Team durchlaufen. Anhand dieses Beispiels sollen die Prozesse evaluiert werden und diese Ergebnisse am Ende mit anderen Erfahrungsberichten verglichen. 