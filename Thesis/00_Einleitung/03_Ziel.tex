\section{Ziel dieser Arbeit}
Viele große Unternehmen haben diese Prozesse für Gründungen und Erweiterungen im eigenen Unternehmen genutzt. Dagegen stehen die Scheiterungsstatistiken von Startups. Daher könnte es auch einfach Zufall sein und jene Unternehmen hätten sich auch ohne diese modernen Methoden gut entwickelt. Es lässt sich keine Aussage darüber treffen, wie viele der Startups am Ende doch scheitern und warum. Doch zusammenfassend kann man sagen, dass der Trend eindeutig in die Richtung geht, kreative Prozesse, wie die in Kapitel \ref{sec:Grundlagen} beschriebenen Konzepte, zu nutzen. Der Mehrwert wird hoch geschätzt, allerdings kommen diese Aussagen hauptsächlich von den Prozessgründern.

Ziel dieser Arbeit ist es, die Auswirkungen von modernen Designprozessen auf den Erfolg von Startup Gründungen zu untersuchen. Genauer werden die Konzepte der \textit{Innovator's Method}, der \textit{Sprint} und \textit{\ac{TLS}} exemplarisch getestet. Aufbauend auf die Ergebnisse des in Abschnitt \ref{sec:Projektbeschreibung} beschriebenen Projektteams und unter Berücksichtigung der von den Autoren beschriebenen Projekte, werden diese Methoden abschließend validiert. 