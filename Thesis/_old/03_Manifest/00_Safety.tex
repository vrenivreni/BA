\section{Anforderungen an das dynamische Netzwerk}\label{sec:NeuesNetz}
An das Fahrzeugnetzwerk werden verschiedenste Anforderungen gestellt. So soll es beliebig erweiterbar im Hinblick auf Skalierbarkeit sein, genügend Bandbreite besitzen, um auch mit großen Datenmengen umgehen zu können, und natürlich sicher vor ungewollten Einflüssen sein. In der Automobilbranche ist Sicherheit ein besonders wichtiger Punkt, weshalb unter der Norm ISO 26262 \cite{ISO26262} die wichtigsten Störungsmöglichkeiten definiert sind.

Generell muss ein Netzwerk alle gängigen Sicherheitsanforderungen gerecht werden. Sie müssen erfüllt sein, um ein Netzwerk nach den sog. Safety-Anforderungen zu erstellen \cite{Safety-critical}. In der Tabelle \ref{tab:Failure} sind die wichtigsten Störungsfälle im System aufgelistet. Anhand dieser Anforderungen, welche als Grundvoraussetzung für sichere System gelten, werden im Folgekapitel die Parameter für das Applikationsmanifest festgelegt. Wenn bereits einige dieser Anforderungen nicht abgedeckt werden, wird es schwierig ein sicheres Netzwerk aufzubauen \cite{FunctionalSafety}.
\begin{table}[!htb]
	\centering
	\begin{tabularx}{0.99\textwidth}{p{0.25\textwidth} | p{0.65\textwidth}}
		Störung & Erläuterung \\
		\hline\hline
		Korrupte Nachricht & Die Daten des empfangenen Paketes einer Nachricht sind falsch \\
		Nachrichten Verzögerung & Nachricht wird später als erwartet empfangen \\
		Nachrichtenverlust & Nachricht oder Paket geht im System verloren \\
		Wiederholte Nachricht & Empfänger erhalten zwei oder mehr Nachrichten mit dem selben Inhalt \\
		Nachfolgeregelung & Nachrichten werden in falscher Reihenfolge empfangen \\
		Unterbrechungen & Empfänger erhalten unerwartet eine Nachrichten, während er noch mit dem Empfangen einer andern Nachricht \newline beschäftigt ist\\
		Maskierung & Nachricht wird unter Verwendung einem gefälschten Identifikationsmerkmal versandt \\
		Asymmetrische \newline Auskünfte & Informationen eines einzelnen Absenders werden von mehreren Empfängern unterschiedlich empfangen \\
		Unsichere \newline Kommunikation & Ermöglichen den unbefugten Zugriff auf \newline Kommunikationsdaten \\
	\end{tabularx}
	\caption[Übersicht der in ISO26262 zertifizierten Störmöglichkeiten]{Übersicht der in ISO26262 zertifizierten Störmöglichkeiten}
	\label{tab:Failure}
\end{table}
