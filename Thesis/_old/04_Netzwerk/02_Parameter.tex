\section{Feature Entwicklung}\label{sec:FeatureDef}
Die im Abschnitt \ref{sec:Netzwerkmanager} Netzwerkmanager angesprochene Features, kümmern sich um die automatisierte Verarbeitung der Parameter aus den Applikationsmanifest und die daraus resultierenden Ableitungen zur Konfiguration des Netzwerkes. In den folgenden Unterabschnitten wir jeder Eintrag aus dem Applikationsmanifest behandelt und definiert, welche Ableitungen von diesem gemacht werden können. Ein sich im Manifest befindender Parameter kann in mehreren Features verarbeitet werden. 
Die Funktionen splitten nicht nur Parameter auf, sie definieren auch Mechanismen, die eine Einhaltung von applikationsspezifischen Anforderungen garantieren. 

Der Aufbau der einzelnen Unterabschnitten erfolgt nach einem vorgegebenen Schema erfolgen:
\begin{itemize}
	\item Definition des Features
	\item Erläuterung durch Verwendungsmöglichkeiten oder Aktivitätsdiagramm 
\end{itemize}
Es wird bei jedem Parameter nur auf die Ableitungen eingegangen, die auch im Automobil-Bereich umgesetzt werden. Wenn bei der Ableitung der Parameter eine bestimmte Technik gefordert wird, die eine \ac{ECU} oder Switch gewährleisten muss, wird der Einfachheit halber auf das Abprüfen ob das Endgerät diese Funktion besitzt im Diagramm verzichtet. Dies wird vor jeder Konfiguration der Geräte sowieso überprüft und liefert eine Abbruchbedingung zurück, falls die Technologie nicht zur Verfügung stehen sollte.

\subsection{Feature Kommunikationsstrecken Aufbau}
Eine Grundvoraussetzung für die Verständigung zwischen Endgeräten ist ein Aufbau einer Kommunikationsstrecke unter den Teilnehmern. Dies bedeutet eine Konfiguration aller Ports die ein Versendete Nachricht passieren muss. Je nach Anforderung aus dem Applikationsmanifest und der übermittelten Informationen aus dem Orchestrator, muss eine geeigneter Weg im System gefunden werden. Die Parameter \emph{id, groupId, domain} und der \acs{MAC}-Adressen der einzelnen Teilnehmer, können verschieden Einstellungen an den \ac{ECU}'s und Switchen vorgenommen werden. Nicht alle haben die selbe Technologie was bedeutet, dass endgerätspezifisch entschieden werden muss, welche Konfigurationen möglich sind. 

Mögliche Verwendungen wären beispielsweise:
\begin{itemize}
	\item Portforwarding, was eine Weiterleitung an einen anderen Port eines Endgerätes bewirkt
	\item Ingress Filtering, ist eine Art Firewall und schützt vor Unerlaubter Kommunikation (Definition in Abschnitt \ref{sec:Ingress})
\end{itemize}

Hauptsächlich dienen die Port-Konfigurationen zum reibungslosen Datenaustausch. Sie können aber auch zu einen gewissen Teil zum einhalten der in Abschnitt \ref{sec:Schutzziele} definierten Schutzziele beitragen. Durch definierter Portregelungen wird ein unerlaubter Datenaustausch erschwert.

\subsection{Feature Echtzeitfähigkeit}
Um das Feature Echtzeitfähigkeit umzusetzen, werden mehre Parameter benötigt. In \emph{maxLatency} ist die geforderte Latenzzeit angegeben, dies ist eine der wichtigsten Eigenschaften für eine Anwendung in einem Netzwerk. Aus dieser können verschieden Einstellungen gefolgert werden. Das zweite Attribut ist die \emph{minLatency}. Dieser ergibt, wie in Abschnitt \ref{sec:RechnungJitter} beschrieben, zusammen mit der \emph{maxLatency} den geforderten Jitter. Ein weiterer Parameter ist \emph{timesync}, dieser wird zur Konfiguration des \acf{TAS} benötigt. Da sich die Anforderung nach \ac{TAS} erst bei der Auswertung der Parameter ergeben kann, muss die Zeitsynchronisation nachträglich konfiguriert werden, falls diese nicht von Anfang an gesetzt ist.

Im folgenden Aktivitätsdiagramm wird grafisch veranschaulicht, wie der Netzwerkmanager anhand der Parameter entscheidet, ob eine Applikation im System laufen kann. Der erste Schritt ist die Prüfung der \emph{maxLatency}. Die Latenz darf nicht kleiner als 100µs sein, da dies physikalisch nur bei einer Punkt-zu-Punkt Verbindung möglich ist . Diese Art der Verbindung bedingt jedoch keine Konfiguration des Netzwerkes, da dies statisch vordefiniert ist. Als nächstes muss die generelle Zeitrelevanz einer Anwendung geprüft werden. Ist die \emph{maxLatency} größer als 200ms, kann die Applikation als zeitunrelevant betrachtet werden. Resultierend daraus, wird lediglich eine Leitung gesucht, bei der das Ankommen der Daten sicherstellt ist. 

Ist Zeitrelevanz jedoch nötig, wird als nächstes geprüft, ob ein Jitter angegeben ist. Dies wird über die \emph{minLatency > 0} geregelt. Ist diese Null, kann an dieser Stelle der Vorgang abgebrochen werden, um das aktuelle Netzwerk zu überprüfen. Im Falle das, das System bereits ausreichend ist, wird schon die gewünschte Bestätigung zurückgegeben. Falls das aktuelle Netzwerk nicht ausreichend ist, weil beispielsweise die Latenz nicht eingehalten werden kann, muss es neu konfiguriert werden. So kann eine Latenz von kleiner 2ms den \acl{CBS} zur Folge haben. Ist eine \emph{minLatency} jedoch gesetzt, bedingt dies den Einsatz eines \acl{TAS}. Welcher wiederum Zeitsynchronisation bedingt.
\newpage
\singlefigure{\label{fig:LatenzDia}}{99_Dias/LatzenzDia.png}{Aktivitätsdiagramm Echtzeitfähigkeit}

\newpage
\subsection{Feature Datenrate}
Die bereits in Abschnitt \ref{sec:Datenrate} erwähnte Datenrate besteht aus zwei Parametern. 
Das folgende Aktivitätsdiagramm zeigt den Ablauf einer Prüfung des aktuellen Netzwerkes. Diese wird anhand der Datenrate aller laufenden Prozesse durchgeführt. Reicht die restliche Bandbreite nicht mehr aus, wird eine Rekonfiguration des Netzwerkes versucht. Erst wenn diese nicht ausreichend ist, wird die Anwendung abgelehnt. In den meisten Fällen wird eine Leitung nie vollständig ausgelastet, um im Falle von \emph{Bursts} keinen Datenverlust zu haben. Als \emph{Bursts} werden Schwankungen in der Datenrate bezeichnet. Dies können zu einer kurzzeitigen Überlastung der Leitung führen.
\singlefigure{\label{fig:DatenrateDia}}{99_Dias/FeatureDatenrate.png}{Aktivitätsdiagramm Datenrate}

\newpage
\subsection{Feature Zeitsynchronisation}
Die in Abschnitt \ref{sec:Zeitsync} erwähnte Zeitsynchronisation ist nicht nur für die Echtzeitfähigkeit wichtig, sonder auch für einen synchronisierten Datenaustausch. Um sicher zu stellen, dass Daten aktuell sind, wird beispielsweise ein gemeinsamer Zeitstempel verwendet. Das Feature Zeitsynchronisation verbindet die Applikation mit einer Master-Clock im System. Ist keine passende Uhr vorhanden, können Mechanismen wie der Best-Master-Clock-Algorithmus angewendet werden \cite{BestMasterClock}. Dieser sucht im System die exaktest Uhr und setzt dies als Master-Clock. Im aktuellen Netzwerk ist die jedoch nur eine Backup-Lösung bei einem Ausfalls einer Master-Clock gedacht, da dies momentan manuell vordefiniert ist. Welche Genauigkeit ein Uhr haben muss, wird durch das Manifest vorgeben und wird lediglich mit ausgelesen. Falls die aktuelle Master Clock im System den Ansprüchen nicht mehr genügt, muss ebenfalls ein neuer Master gesucht werden. Das Diagramm zeigt den Ablauf der Findung einer neuen Master Clock. 
\singlefigure{\label{fig:TimeSync}}{99_Dias/FeatureTimeSync.png}{Aktivitätsdiagramm Zeitsynchronisation}

\newpage
\subsection{Feature Authentizität}
Das Feature Integrität soll den unbefugten Zugriff auf kritischer Daten verhindern. Im Manifest ist es in Form eines Flags modelliert. Der Wert in \emph{integrity} entspricht einem boolischen Wert im Bereich 0 und 1. Dies ist gleichzusetzen mit einem ein- und ausschalten der Funktionalität. Falls das Flag duch eine 1 gesetzt ist, werden Sicherheitsmechanismen wie \ac{MACsec} (Abschnitt \ref{sec:MacSec}) im System konfiguriert. So wird \ac{MACsec} bei allen Teilnehmern der Kommunikationsstrecke an den jeweils verwendeten Port konfiguriert. Es gibt jedoch Endgeräte die die Technologie noch nicht unterstützen, dies muss bei der Konfiguration beachtet werden. Im nachfolgen Diagramm wird der Prozesses für alle beteiligten Netzwerkgeräte dargestellt.
\singlefigure{\label{fig:Integrity}}{99_Dias/integrity.png}{Aktivitätsdiagramm Integrität}