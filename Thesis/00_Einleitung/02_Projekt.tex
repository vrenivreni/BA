\section{Startup Projekt}
%Das Projekt A3F wurde im Jahr 2016 von der Continental Automotive GmbH und der OTH Regensburg gegründet. Das Hauptforschungsthema liegt in der Entwicklung einer ausfallsichere\add{n} Architektur\remove{en} für autonome Fahrzeuge. Anhand von Schlüsselszenarien sollen die Herausforderungen beim Übergang hin zu einer neuen flexiblen und dynamischen Fahrzeugarchitektur herausgearbeitet werden. Hierbei geht es darum, die Anforderungen auf Applikationsebene zu verstehen, um daraus dann die Technologien und Lösungskonzepte für das Netzwerk im Fahrzeug der Zukunft abzuleiten. Ziel ist es, unter dem Gesichtspunkt der Wiederverwendung vorhandener Methoden und Technologien aus der Enterprise IT zu beleuchten und diese hinsichtlich eines möglichen Einsatzes im Fahrzeug zu bewerten. Hierzu müssen diese Methoden und Technologien neben den funktionalen Eigenschaften auch noch i\change{n}{m} Hinblick auf die speziellen automotive Anforderungen betrachtet werden. Darüber hinaus sollen erste Ansätze zu einem möglichen Migrationspfad, von bestehenden zu zukünftigen, an automotiven Anforderungen angepasste Lösungen aufgezeigt werden. Am Ende des Projekts sollen die gewonnenen Ergebnisse anhand eines Demonstrators gezeigt werden.
%Daraus lassen sich folgende drei globale Ziele definieren:
%\newpage
%\begin{itemize}
%	\item Identifizierung von Schlüsseltechnologien und Methoden für zukünftige Fahrzeug-Architekturen
%	\item Daraus Empfehlungen für zukünftige Kommunikationsnetzwerke im Fahrzeug ableiten
%	\item Transfer von potentiellen IT-Technologien ins Fahrzeug unter Berücksichtigung von automotive Anforderungen
%\end{itemize}