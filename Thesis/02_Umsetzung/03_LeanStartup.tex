\section{The Lean Startup}
Die grundlegende Methodik des \ac{TLS}-Konzeptes wird in Abschnitt \ref{sec:LeanStartup} beschrieben, wonach das Agrishare-Team einzelne Themen bearbeitet hat. Die Umsetzung, sowie Validierung folgt im weiteren Verlauf dieses Kapitels.
\subsection*{\refstepcounter{subsection}\label{sec:LeanStartup-Umsetzung-ValidatedLearning}\thesubsection\quad Validated Learning}
Bereits zu Beginn des Projektes wird bei Agrishare darauf geachtet, jegliche Hypothesen über das Nutzungsverhalten zu überprüfen. 

So steht beispielsweise die These im Raum, dass Landwirte aktuell ihre Arbeiten über einen Großkonzern vermitteln, dann aber eigenständig abrechnen, um sich die Zuschläge für diesen Konzern zu sparen. Überraschenderweise stellt sich heraus, dass genau das Gegenteil zutrifft. Meist kennen sich Landwirte ohnehin untereinander und wissen, welcher Bauer in der Nähe welche Maschinen besitzt und welche Arbeiten verrichten kann. Das heißt, die Vermittlung geschieht eigenständig. Jedoch erfolgt die Abrechnung ausschließlich über oben genannten Konzern, da dieser für einwandfreie Rechnungsstellung und Dieselzertifikate sorgt. Das wiederum verringert den bürokratischen Aufwand für die Landwirte.

Außerdem wird das Konzept, Maschinen ausschließlich versichert zu vermieten, anfänglich als sinnvoll angesehen. Nach einigen Gesprächen mit potentiellen Endkunden wird dies jedoch verworfen. Der Grund dafür ist, dass Landwirte ohnehin eigenständig sind und daher über ausreichend Versicherungsschutz verfügen müssen, wie einige Landwirte behaupten. Zusätzliche Absicherung würde daher keinen Wert für den Endnutzer bringen.

\subsection*{\refstepcounter{subsection}\label{sec:LeanStartup-Umsetzung-InnovationAccounting}\thesubsection\quad Innovation Accounting}
Nach dem \textit{Innovation Accounting} Konzept wird zuerst ein MVP entwickelt, um den Ausgangszustand festzusetzen. Das \ac{MVP} entsteht aus dem zuvor durchgeführten Sprint Prozess.

In der ersten \ac{BML} Iteration wird auf der Startseite ein Abschnitt eingefügt, der dem Nutzer personalisierte Vorschläge zeigt. Angelehnt an persönliche Interessen, das frühere Nutzungsverhalten und den aktuellen Standort des Kunden werden attraktive Angebote für Maschinen oder Dienstleistungen angezeigt. Der Nutzertest hat hier eindeutig gezeigt, dass das Feature wertvoll ist. 

\subsection*{\refstepcounter{subsection}\label{sec:LeanStartup-Umsetzung-Pivot}\thesubsection\quad Pivot}Erste Probleme entstehen in der nächsten Iteration. Zunächst ist die Plattform aufgeteilt in die zwei Blöcke \textit{Vermittlung} und \textit{Transaktion}. Dabei ist der Transaktionsteil besonders kritisch, da darauf geachtet werden muss, dass der Geldfluss problemlos vonstatten geht und auch die Rechnungsstellung einwandfrei funktioniert. Nachdem das Team viel Zeit in die Detailplanung der Abrechnung investiert hat, fällt im Nutzertest auf, dass dieser Teil der Anwendung nicht unbedingt erforderlich ist. Nach diesen neuen Erkenntnissen entscheidet das Agrishare Team, ein \textit{Pivot} durchzuführen. Dieses kann am besten durch ein \textit{Zoom-In Pivot} beschrieben werden. Denn das Produkt wird auf die Vermittlung eingeschränkt, was ursprünglich nur einen Teil des Gesamtproduktes dargestellt hat. Diese Entscheidung stimmt das Team sehr positiv, da dadurch eine große Verantwortung und damit eines der größten Risiken wegfällt. 

\subsection*{\refstepcounter{subsection}\label{sec:LeanStartup-Umsetzung-Batches}\thesubsection\quad Batches}Das Batch-Konzept wird im Agrishare Team ebenfalls angewandt. So wird das Backend auf Abruf entwickelt und in enger Zusammenarbeit mit den Fortschritten im Frontend angepasst und optimiert. Im Gegensatz dazu wäre es auch möglich gewesen, Frontend und Backend separat voneinander zu entwickeln und im Nachhinein zusammenzufügen. Wie \citeauthor{Sprint} jedoch voraussagt, würden dadurch große Inkompatibilitäten entstehen. Dadurch müsste sehr viel zusätzliche Arbeit investiert werden, die beiden fertigen Blocks aufeinander abzustimmen. Durch kleinere Blöcke, welche regelmäßig zusammengeführt und aufeinander abgestimmt werden, wird dieser Arbeitsaufwand minimiert.

\subsection*{\refstepcounter{subsection}\label{sec:LeanStartup-Umsetzung-EngineOfGrowth}\thesubsection\quad Engine of Growth}Das Wachstum des Unternehmens soll mit dem \textit{viral engine of growth} gemessen werden. Diese Methode erweist sich hier als sinnvoll, da eine Plattform davon lebt, viele Nutzer in enger räumlicher Dichte zu haben. Außerdem macht es Sinn, die bereits vorhandene gute Vernetzung der Landwirte auszunutzen. Dies hätte zur Folge, dass durch Mundpropaganda die Plattform an Nutzern zunehmen würde. Das ist allerdings eine Prognose, die so nicht im Vorhinein überprüfbar ist. Daher kann die Unsicherheit, die \citeauthor{Sprint} dazu treibt, das \ac{TLS}-Konzept zu entwickeln, nie komplett aus dem Weg geräumt werden.

\subsection*{\refstepcounter{subsection}\label{sec:LeanStartup-Umsetzung-Evaluierung}\thesubsection\quad Endresümee - The Lean Startup}Nach den Aussagen des Agrishare Teams ist das Konzept des \ac{TLS} weitestgehend sinnvoll. Allerdings besteht es aus Methoden, die logisch scheinen und nicht überraschend sind. Das heißt, es wird ohnehin angenommen, dass jedes Startup, das sich ansatzweise mit Herangehensweisen an Unternehmensgründung auseinandersetzt, diese oder ähnliche Methoden anwenden. Objektiv betrachtet, ist es einfach, solche Aussagen zu treffen, nachdem sich bereits mit dem Prozess auseinandergesetzt wurde. Das Team kann daher keine Aussagen darüber treffen, wie diese Dinge angegangen worden wären, wäre \ac{TLS} nicht bekannt gewesen. Anhand der Beispiele, welche \citeauthor{Sprint} \citeyear{Sprint} in seinem Buch beschreibt, werden diese Dinge in der Praxis selten angewandt. Allein durch diese Fehler, kam er letztendlich auf die Idee, \ac{TLS} zu entwickeln.


