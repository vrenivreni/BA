\section{Netzwerkmanagement im Automotiv-Bereich}
Dieser Abschnitt erklärt den aktuellen und zukünftigen Stand der Technik in einem moderneren Fahrzeug.  

Da die Größe der zu sendenden Daten in einem Fahrzeug ständig weiter steigt, ist es notwendig eine neue Generation von fahrzeug-interner Netzwerkinfrastruktur zu entwickeln. Die bisher genutzten Techniken, wie \acf{CAN}, \acf{LIN} und FlexRay sind für diese Datenmengen nicht mehr ausreichend. Aus diesem Grund versucht man Ethernet in das Fahrzeug zu integrieren, was in manchen Anwendungsbereichen bereits der Fall ist. Ethernet wurde bereits unabhängig von der Automobilbranche unter der IEEE 802.3 \cite{IEEE802_3} standardisiert und ist ein vielversprechender Ansatz, da es sich durch eine hohe Wiederverwendbarkeit für Komponenten, Software und Tools auszeichnet. Ebenso verfügt es über eine ausreichend dimensionierte Bandbreite und gute Skalierbarkeit des gesamten Netzwerkes. Der Hauptvorteil von Ethernet ist, dass jeder Teilnehmer seine eigene Leitung besitzt und anders als bei den Bus-Systemen kein geteiltes (eng. \emph{shared}) System ist. Hierbei nutzen alle Teilnehmer dieselbe Leitung. Im Automobilbereich spielen aber auch noch andere Kriterien eine Rolle, wie beispielsweise Kosten, Robustheit und der Stromverbrauch \cite{AutomotiveApplications}. Anhand dieser Kriterien muss sich Ethernet erst noch gegen seine bereits im Fahrzeug fest etablierten Konkurrenten durchsetzen. Diese Systeme erfüllen die benötigten Echtzeit-Anforderungen, wie eine garantierte Latenzzeit der Datenpakete, an eine Kommunikation zwischen den Endgeräten. Da Ethernet jedoch anders als Bus-Systeme ein verbindungsloses Netzwerkmedium ist, muss auf eine andere Weise sichergestellt werden, dass Daten in Echtzeit ankommen. Dies kann mittels \acf{TSN} umgesetzt werden, welches im Abschnitt \ref{ref:TSNGrundlagen} näher erläutert wird.


%In der nachfolgenden Tabelle \ref{tab:bus} werden aktuelle und geplante Standardisierungen im Bereich Netzwerktechnologien aufgezeigt. Es werden Eckdaten zu Bandbreite, Erscheinungsjahr und Normen angegeben, unter denen weiter Informationen erlangt werden können. Unter Zuhilfenahme der einzelnen Standardisierungen, wird versucht ein möglichst effizientes Netzwerk zu erstellen. 
%
%\begin{table}[!htb]
%	\centering
%	\begin{tabularx}{\textwidth}{p{0.25\textwidth}|p{0.20\textwidth}|p{0.20\textwidth}|X}
%		Standard & Bandbreite & Name & Veröffentlichung\\
%		\hline\hline
%		ISO-Norm 17987-1 & 25kbit/s & LIN BUS & 1998\\
%		ISO 11	898 & 1Mbit/s & CAN BUS & 1986\\
%		IEEE 802.3i & 10Mbit/s & 10Base-T & 1990 \\
%		ISO 17458-1/5 & 10Mbit/s & FlexRay & 2000\\
%		IEEE 802.3u & 100Mbit/s & 100Base-TX & 1995\\
%		IEEE 802.3ab & 1000Base-T & 1Gbit/s & 1999\\
%		IEEE 802.3an & 10Gbit/s & 10GBase-T & 2002\\
%		IEEE 802.3ba & 100Gbit/s & 100GBase & 2010\\
%		\multirow{2}{*}{IEEE 802.3bs}
%		& 200Gbit/s & 200GBase & 2017 \\
%		& 400Gbit/s & 400GBase & 2017 \\
%	\end{tabularx}
%	\caption[Aktuelle Netzwerkstandards in modernen Fahrzeugen]{Aktuelle Standards in modernen Fahrzeugen}
%	\label{tab:bus}
%\end{table}

