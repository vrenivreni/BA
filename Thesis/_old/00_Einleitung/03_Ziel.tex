\section{Ziel dieser Arbeit}
Wie bereits erwähnt ist das Ziel ein Fahrzeugnetzwerk zu entwickeln, welches durch funktionelle Applikationen erweitert werden kann. Hierzu sollte das Konzept von Plug and Play verwendet werden, d.h. verschiedene Anwendungen müssen aus einem zentralen Anwendungsverzeichnis geladen werden können und ohne manuelles Eingreifen in das Fahrzeugnetzwerk lauffähig sein. 

Die automatische Konfiguration soll durch ein Applikationsmanifest ermöglicht werden. Es gibt die Randbedingungen an, welche zum reibungslosen Betrieb der Anwendung benötigt werden. Das Manifest soll bereits vom Entwickler der Anwendung, anhand der spezifischen Anforderung an die Datenübertragung im Netzwerk eines Fahrzeuges, festgelegt werden. Da viel der neu entwickelten Anwendungen nicht die gleichen Ansprüche an das Netzwerk stellen, ist es wichtig bereits bei der Entwicklung an die nötigten Parameter für die Anwendung zu denken. Basierend auf diesen Parametern, wird eine bereits bestehende zentrale Netzwerkkontrolleinheit, der sog. Netzwerkmanager, funktionell weiterentwickelt. Dieser soll in Zusammenarbeit mit weiteren Komponenten im System eine automatische Konfiguration des Netzwerkes vornehmen.

Die Veranschaulichung des Anwendungsfalles der Arbeit ist in Abb.\ref{fig:EinManager} dargestellt. Diese zeigt den Netzwerkmanager, hier zur Vereinfachung als Blackbox dargestellt. Anhand der Eingabeparameter aus dem Manifest allein können noch keine Netzwerkkonfigurationen erstellt werden. Der Manager muss auch über die Eigenschaften der Netzwerkteilnehmer Bescheid wissen. Nicht jedes Gerät kann die geforderten Bedingungen mit den dafür erforderlichen Technologien umsetzen. Aus diesem Grund sendet jede beteiligte Komponente seine Eigenschaften ebenfalls an den Netzwerkmanager. Dieser wertet sein Input automatisch aus und erstellt eine Konfiguration für alle Teilnehmer der Kommunikationsstrecke im Fahrzeugnetzwerk.

\singlefigure{\label{fig:EinManager}}{99_IMG/01_Einleitung/manager.png}{Vereinfachte Darstellung der Netzwerkkonfiguration}

Im Laufe der Arbeit werden Endgeräte als \acf{ECU} bezeichnet. Netzwerkgeräte sind Switche und andere vermittelnde Komponenten im System.

Die Ausarbeitung des Themas erfolgt nicht nur theoretisch, sondern wird teilweise praktisch umgesetzt. Um dies zu ermöglichen müssen mehrere verschiedene Features implementiert werden. Der Praktische Teil beschränkt sich jedoch nur auf ein Fallbeispiel, da dies sonst den zeitlichen Rahmen der Arbeit sprengen würde. Anhand des Testfalles soll die Funktionalität der automatischen Konfiguration mittels eines Applikationsmanifestes belegt werden.