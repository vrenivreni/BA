\chapter{Schlusswort}
\section{Zusammenfassung}
Nachdem sich Agrishare im Rahmen des Hackaburgs bildet, steht bereits ein grobes Konzept. Demnach soll eine Online-Plattform die Ressourcenauslastung in der Agrarbranche effizienter gestalten. Nach dem Erfolg bei dem Event beschließt das Team, ein Startup zu gründen, um die Plattform auf den Markt zu bringen. Nach gründlicher Recherche zur Gründung eines IT-Unternehmens im Allgemeinen, stößt die Gruppe auf eine Reihe erfolgsversprechender Konzepte. So auch die in Kapitel \ref{sec:Grundlagen} beschriebenen Prozesse. Aufbauend auf die Innovator's Method in Kapitel \ref{TheInnovatorsMethod} entstehen bereits wertvolle Einblicke in die Kundensicht. So wird klar, dass in der Agrarbranche ein Weg, schnell und effizient Ressourcen zu vermitteln, benötigt wird. Mit diesem Wissen startet das Team einen fünftägigen Workshop nach dem Prinzip des Sprints von Eric Ries, wie in Kapitel \ref{sec:Sprint-Grundlagen} zusammengefasst. Die Umsetzung ist in Kapitel \ref{sec:Sprint-Umsetzung} genauer beschrieben. Allerdings wird der Prozess etwa ab der Hälfte dahingehend abgewandelt, dass die Entwicklung des Prototypen verlängert wird. Der Grund dafür ist, dass dies den ersten Prototypen auf Webbasis darstellt. Daher ist es wichtig, dass die einzelnen Bestandteile einheitlich aufgebaut sind, was nicht innerhalb der vorgegebenen Zeit realisierbar ist. Trotzdem schafft es das Team, durch den Sprint viele Details im Vorhinein abzuklären. So können die unterschiedlichen Ansprüche der Mitglieder an das Produkt, welche überraschenderweise sehr voneinander abweichen, miteinander vereinbart werden. Außerdem werden wichtige Begriffe eindeutig definiert, sodass diese zukünftig nicht für Verwirrungen im Team sorgen. Viele andere Punkte, welche unter Punkt \ref{sec:Sprint-Umsetzung} genauer erklärt sind, sorgen für ein durchgehend positives Feedback des Teams. Der Workshop wird im Allgemeinen als sehr hilfreich angesehen, auch wenn er abgeändert werden muss. Mit den Erkenntnissen aus dem Sprint und den anschließenden Tests können Konzepte der Lean-Startup-Methode angewandt werden.

\section{Fazit und Ausblick}
Abschließend lässt sich sagen, dass die Kundenorientierung, welche alle genauer getesteten Konzepte gemeinsam haben, durchaus sinnvoll ist. Das kann auch an oben beschriebenem Beispiel belegt werden. So hätte das Team vermutlich viel Zeit in die Umsetzung der Transaktion investiert, welche sich als unnötig herausgestellt hat. Durch die Lean Startup Methoden konnte dies im Vorhinein eliminiert werden.