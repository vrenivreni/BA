\section{Sprint - Google Ventures}
\label{sec:Sprint-Grundlagen}
Der Google Ventures Sprint soll innerhalb von fünf Tagen Teams dabei helfen, ein neues Produkt oder eine Erweiterung zu definieren und mit potentiellen Endkunden zu testen. Besonders empfohlen wird dieser Prozess, wenn ein Team vor einer immens großen Aufgabe oder kurz vor einer Deadline steht, oder auch wenn ein Punkt erreicht wurde, an dem man nicht weiter kommt. Für diesen Sprint wird nur das Team, welches an dem Projekt arbeiten soll, und ein Raum mit vielen Whiteboards benötigt. Dabei besteht die größte Schwierigkeit darin, bei Startups sowie in etablierten Unternehmen, allen Teammitgliedern fünf komplette Tage freizuhalten. Damit der Workshop möglichst produktiv ist, sollten alle Teilnehmer auf ihre elektronischen Geräte nach Möglichkeit verzichten. Die optimale Teilnehmeranzahl für einen Sprint besteht aus maximal sieben Personen. Dabei kann es von Vorteil sein, eine Mischung an verschiedenen Personalitäten einzuladen. Das können Personen sein, die in unterschiedlichen Branchen arbeiten oder generell alternativ denken und handeln. Darüber hinaus kann es auch nützlich sein, Experten für speziellere Bereichen aus einem Projekt einzuladen. Dazu zählen beispielsweise Finanz-, Marketing-, Kunden-, Technik- oder Designexperten. Außerdem sollte eine Person die Rolle des Deciders übernehmen. Dieser ist meist der Teamleiter oder Abteilungsleiter und sollte das größte Know-How über das Projekt haben. Weitere Spezialisten können auch nur für die Expertenrunden an Tag 1 eingeladen werden, falls die Teilnehmeranzahl sonst überschritten wird. Außerdem wird ein Facilitator benötigt, welcher den Ablauf des Workshops kennt und regelt, sowie sicherstellt, dass das Team konzentriert arbeitet und in Diskussionen nicht abschweift. Wenn das Team aufgestellt und der Raum vorbereitet ist, kann der Sprint beginnen. Der Ablauf wird in den folgenden Absätzen genauer erläutert. Die Absätze sind in Tage unterteilt, um den Verlauf des Sprints besser darstellen zu können.

\subsection*{Tag 1}
\paragraph{Long Term Goal}
Der Sprint beginnt mit der Festlegung eines Langzeitzieles. Darunter versteht man einen Satz, der aussagt, was mit dem Projekt auf lange Sicht erreicht werden soll. In einer Gruppendiskussion werden hier Fragen wie \textit{Warum wird das Projekt durchgeführt? Wo sehen wir uns in 6 Monaten, in einem Jahr oder sogar in fünf Jahren?} geklärt. Oft kann es hier dazu kommen, dass die Teammitglieder unterschiedliche Erwartungen haben. Diese Aufgabe ist also einerseits dafür gedacht, die Erwartungen des ganzen Teams zu einer realistischen Gesamterwartung zusammenzufassen. Andererseits soll das Langzeitziel auch während des kompletten Sprints dazu dienen, den Fokus nicht zu verlieren. Da dieser Satz immer wieder ins Gedächtnis gerufen wird, ist es einfacher, konzentriert auf dieses Ziel hinzuarbeiten. Damit dieser Satz den kompletten Sprint über verfügbar ist, wird er gut sichtbar auf ein Whiteboard notiert. 

\paragraph{Sprint-Fragen}
Nachdem die Erwatungen aller Teilnehmer zusammengefasst und geklärt wurden, werden im nächsten Schritt alle Bedenken gesammelt. Dafür werden offene Fragen aus der Runde auf ein weiteres Whiteboard geschrieben. Jeder Teilnehmer hat nun die Aufgabe, seine Bedenken oder Ängste zu einer Frage zu formulieren oder auch Fragen, die er gerne im Laufe der Zeit beantwortet haben möchte, zu stellen. Hierbei wird nicht aussortiert oder geurteilt, alle Beiträge aus dem Team werden notiert und somit auch über den Sprint-Zeitraum im Gedächtnis behalten. 

\paragraph{Map}
Die nächste Aufgabe besteht darin, das Gesamtprojekt auf einer Karte oder einem Graphen abzubilden. Dieser Darstellung soll das Projekt möglichst einfach aber trotzdem verständlich darlegen. Festgehalten wird diese Karte auf dem ersten Whiteboard direkt unter dem Long Term Goal. Dabei wird mit den Schlüsselpersonen begonnen, welche auf der linken Seite untereinander gelistet werden. Darunter versteht man meist unterschiedliche Kundengruppen oder auch andere wichtige Personengruppen, wie beispielsweise der Staat. Danach wird rechts das Ende des Gesamtprozesses aufgeschrieben. Das besteht aus einem Stichwort, welches in gewisser Weise auch den Sinn des ganzen Projekts beschreibt, wie zum Beispiel \textit{Behandlung} oder \textit{Kauf}. Nachdem Start und Ziel des Produktablaufes festgehalten sind, werden nun die Zwischenschritte eingetragen. Das beinhaltet jeden Schritt, den eine Schlüsselperson durchlaufen muss, um das Ziel auf der rechten Seite zu erreichen. Hierbei ist es essentiell, diese zwar abstrakt zu halten, aber trotzdem die wichtigsten Schritte zu erkennen und festzuhalten. Als Richtwert sollten am Ende zwischen fünf und 15 Schritte dargestellt sein.

\paragraph{Ask the Experts}
Für diese Aufgabe sollten im Vorfeld Personen eingeladen werden, welche sich in verschiedenen Teilbereichen des Gesamtprojekts besonders gut auskennen. Diese können auch aus dem Sprint-Team selbst sein oder extern für diesen Nachmittag dazustoßen. Das hat den Hintergrund, dass sich selbst die Projektleiter nicht in jedem Teilgebiet gut genug auskennen, um detaillierte Beschreibungen und Erklärungen zu liefern. Welche Experten eingeladen werden sollten, ist immer abhängig von der Komplexität des Projekts. Allerdings gibt es hierfür grobe Vorschläge von Google Ventures.
Die Gebiete, aus denen man idealerweise einen Fachmann zu Rate ziehen sollte, sind folgende: 
\begin{itemize}
	\item \textbf{Strategie:}
	Es wird hier durchaus empfohlen, jemanden, bespielsweise den Decider, über die Gesamtstrategie vortragen zu lassen. Das dient dem Allgemeinverständnis des Teams über das Ziel und die Schritte zum Erreichen dieses Ziels. 
	\item \textbf{Kundensicht:}
	Diese Person sollte ein Mitarbeiter sein, welcher am meisten von der Kundenseite versteht. Dieser sollte auch in der Lage sein, die Sichtweise der Kunden präzise zu erklären und mögliche Risiken oder interessante Einsichten aufbringen.
	\item \textbf{Produktexperte:}
	Im Gegensatz zur Kundenseite ist es außerdem wichtig, jemanden einzuladen, der die Produktseite vertritt. Dieser sollte die Funktionalität des Endproduktes kennen und über die Entstehung des Produkts Bescheid wissen. Da es in diesem Feld sehr viele Einzelbereiche gibt, wie zum Beispiel Finanzen, Technik/Logistik oder Marketing, ist es üblich, mehrere verschiedene Experten einzuladen. Dabei geht es vor allem darum, herauszufinden, wie dieses Wissen zusammenpassen kann. Die Art und Anzahl der Experten hängt wiederum vom Projekt ab und kann deshalb nicht pauschal eingegrenzt werden.
	\item \textbf{Probleme in der Vergangenheit:}
	Eventuell gab es in der Vergangenheit bereits einzelne Personen oder Teams, welche sich intensiver mit dem Thema des Projekts befasst haben. Dann könnten diese darüber berichten, welche Probleme aufgekommen sind oder ob es bereits Lösungsansätze gibt.
\end{itemize}

Um den zeitlichen Rahmen des Workshops einzuhalten, sollte man diese Expertenrunden auf 30 Minuten pro Person eingrenzen. Falls der Fachmann nicht Teil des Sprint-Teams ist, wird zuerst der Hintergrund des Sprints erläutert und die Whiteboards mit dem Sprint-Ziel, der Map und den Fragen kurz nähergebracht. Danach erzählt der Experte frei über sein Spezialgebiet im Bezug auf das Projekt. Das Team stellt dabei viele Fragen, sodass jeder Teilnehmer ein ausreichend spezifiziertes Verständnis für diesen Bereich bekommt. Falls nötig, werden nun die Whiteboards abgeändert. Das heißt, falls sich Neuerungen ergeben, welche nicht mit dem Ziel oder der Map vereinbar sind, dürfen diese an der Stelle angepasst werden. Außerdem sollten wichtige Fragen bei den Sprint-Fragen hinzugefügt werden.

Während der Expertenrunde soll jeder Teilnehmer nicht nur Fragen stellen, sondern auch Notizen machen. Dafür wird eine Vorgehensweise vorgeschlagen, welche sich \textit{How Might We} nennt. Dabei bekommt jede Person einen Block mit Haftnotizen und einen schwarzen Marker. In die linke obere Ecke werden die Buchstaben HMW geschrieben, um den Fragesatz einzuleiten. Wenn man etwas Relevantes aufschnappt, wird diese Information als Frage formuliert auf die Haftnotiz geschrieben und diese Notiz beiseite gelegt.

\paragraph{Notizen organisieren}
Nachdem die Expertenrunden abgeschlossen sind, werden die Haftnotizen von allen Teammitgliedern unstrukturiert an eine freie Wand geklebt. Dann ordnet das Team die Notizen zu logischen Gruppen und findet eine passende Überschrift für die jeweiligen Kategorien. Typischerweise können die restlichen Notizen ohne logischen Zusammenhang zu der Gruppe \textit{Sonstige} zusammengefasst werden.
Die geordneten Memos sollen nun priorisiert werden. Dafür bekommt jedes Teammitglied zwei Sticker, nur der Decider erhält vier. Nachdem sich alle das Gesamtziel und die Fragen erneut vor Augen gerufen haben, kleben sie die Sticker still auf jene Notizen, welche sie für am Wichtigsten erachten. Dabei darf auch die eigene gewählt werden und es dürfen auf einen Zettel auch mehrere Sticker geklebt werden. Nach der stillen Abstimmung werden die Notizen mit den meisten Aufklebern zu den logisch passenden Schritten auf der Map geklebt.

\paragraph{Fokus des Sprints}
Zum Abschluss des ersten Sprint-Tages wird der Fokus für die restlichen Tage festgelegt. Dieser Fokus wird alleine vom Decider bestimmt, allerdings darf er das Team um Hilfe bitten. Dafür schreibt jedes Teammitglied die Schritte der Map auf, die für ihn am wichtigsten sind. Die Vorschläge werden daraufhin auf einem Whiteboard gesammelt und kurz diskutiert, falls es stark abweichende Meinungen gibt. Nachdem der Decider durch die Vorschläge einiges an Input sammeln konnte, muss dieser nun auf der Map eine Zielgruppe und einen Zielschritt einkreisen. Danach werden wiederum die Sprint-Fragen wiederholt und jede markiert, die nach der Fokusauswahl im Rahmen des Sprints als lösbar scheinen.

\subsection*{Tag 2}
Der Vormittag des zweiten Tages läuft unter dem Motto \textit{Remix and Improve}. Es geht darum, bewährte Methoden oder Teile anderer Produkte auf das Projekt zuzuschneiden oder anzupassen. Wichtig dabei ist es, auch innerhalb anderer Branchen zu suchen und kreativ zu sein. Meist ist es nicht auf den ersten Blick erkennbar, wie andere Produkte auf das eigene Projekt angepasst werden können. Es wird oft übersehen, dass es womöglich bereits bestehende Technologien gib, auf die zurück gegriffen werden kann.

\paragraph{Lighting Demos}
Hier werden diese Technologien eingehend untersucht und anschließend vorgestellt. Dafür wird zuerst eine Liste mit Produkten erstellt, die Parallelen zu dem zu entwickelnden Produkt enthalten. Jedes Teammitglied macht sich dazu Gedanken und liefert Vorschläge. Die Liste soll außerdem auch Vorschläge aus anderen Branchen enthalten.

Nachdem die Liste fertiggestellt wurde, stellt jeder Teilnehmer sein(e) Produkt(e) innerhalb von drei Minuten vor. Dabei geht die Person hauptsächlich auf die für das Projekt relevante Bestandteile ein. Damit die Demonstration für alle gut sichtbar ist, kann es ratsam sein, die Produkte mittels Beamer für alle sichtbar zu präsentieren. Der Facilitator zeichnet währenddessen die besonders guten Ideen an ein weiteres Whiteboard, gibt jeder Skizze eine Überschrift und notiert die Quelle darunter. So ist es für das gesamte Team einfacher, sich auf die Demos zu konzentrieren, da sichergestellt ist, dass die Ideen festgehalten und so nicht vergessen werden.

\paragraph{Aufgabenverteilung}
Nun werden die einzelnen Bauteile des Prototypen auf die Teilnehmer aufgeteilt. Falls der Prototyp nur aus einem einzigen Teil besteht, kann auch das gesamte Team an der gleichen Sache arbeiten. Ist das nicht der Fall, sucht sich zunächst jeder Teilnehmer aus, woran er gerne arbeiten würde. Ist die Aufteilung ungerecht, finden sich optimalerweise freiwillige Wechsler. Ansonsten verteilt der Facilitator die Aufgaben auf die Teammitglieder.

\paragraph{Sketch}
Sketching bedeutet, eine Idee auf einem großen Bogen Papier aufzuzeichnen. Das heißt, der gesamte Nachmittag des zweiten Tages besteht daraus, dass jede Person einen individuellen Prototypen zeichnet. Die Skizze soll möglichst ohne schriftliche Erklärungen auskommen und gut verständlich sein. Im Allgemeinen ist es ohnehin einfacher, abstrakte Ideen durch Skizzen zu erklären, als durch Worte. Diese Aufgabe wird wiederum alleine durchgeführt. Dadurch hat jedes Teammitglied die Chance, sich selbst Inspiration zu holen und in der Tiefe über das Problem und eine geeignete Lösung nachzudenken. Damit jedes Teammitglied konzentriert arbeitet und nicht abgelenkt wird, ist die Aufgabe in kleinere Teile aufgeteilt. In den folgenden vier Schritten entwickelt nun jede Person einen eigenen Sketch:
\begin{enumerate}
	\item \textbf{Notes:} 20 Minuten
	
	In Schritt 1 geht jeder Teilnehmer durch den Sprint-Raum und schreibt zuerst das Ziel auf ein Stück Papier. Dann sammelt dieser weitere Notizen indem er sich die Sprint-Fragen, die Map und alle gesammelten Mitschriften erneut vor Augen führt. Außerdem ist es hier auch erlaubt, Smartphones oder Laptops zu benutzen, um sich Inspiration oder zusätzliche Informationen zu holen. Innerhalb der letzten drei Minuten werden die wichtigsten Notizen dann markiert.
	\item \textbf{Ideas:} 20 Minuten
	
	Dieser Schritt dient dazu, eigene Ideen zu entwickeln. Es sollen möglichst viele unterschiedliche Skizzen entstehen. Dazu zählen zum Beispiel kleine Zeichnungen, Beispielüberschriften, Diagramme, Strichfiguren oder Ähnliches. Es geht hauptsächlich darum, kreativ zu sein und viele unterschiedliche Ansätze zu schaffen. Am Ende werden wiederum die besten Skizzen markiert.
	\item \textbf{Crazy 8s:} 8 Minuten
	
	Diese Aufgabe 	besteht daraus, innerhalb von 8 Minuten die beste Idee aus dem Schritt vorher in 8 unterschiedlichen Variationen zu skizzieren. Hierbei bleibt nicht viel Zeit für große Überlegungen, daher entstehen optimalerweise viele spontane Ideen.
	\item \textbf{Lösungs-Sketch:} 30+ Minuten
	
	Im letzten Schritt erstellt jedes Teammitglied den finalen Sketch. Jede Person sucht zunächst die beste Idee aus den vorherigen Schritten aus und versucht, diese auszuarbeiten. Diese Skizzen haben das Format eines dreistufigen Storyboards, denn ein Produkt besteht nie aus nur einem Bild. Stattdessen interagiert der Kunde mit dem Produkt, was immer in mehreren Einzelschritten erfolgt. Falls der Sprint-Fokus so eingeschränkt ist, dass es sich tatsächlich nur um einen kleinen Teil des Gesamtprojektes handelt und es daher sinnvoller ist, sich auf eine Seite zu beschränken, kann von dem Storyboard auch abgewichen werden. Bestenfalls sollten die Sketche selbsterklärend und anonym sein. Dabei darf es sich um ganz einfache Zeichnungen handeln, allerdings ist die Wortwahl äußerst wichtig. Am Ende hat jede Zeichnung außerdem einen eindringlichen Titel.
\end{enumerate}

\subsection*{Tag 3}
Am dritten Tag wird eine Entscheidung darüber getroffen, welcher Lösungssketch am folgenden Tag prototypisch umgesetzt wird. Dafür gibt es viele einzelne Schritte, um zu vermeiden, dass endlose Diskussionen hervorgerufen werden oder die Person mit dem größten Überzeugungstalent den eigenen Sketch am besten verkauft.

\paragraph{Kunstmuseum}
Zuerst werden alle Sketche nebeneinander an einer Wand angebracht. Dabei soll zwischen den Skizzen noch etwas Platz frei sein, ähnlich wie in einem Kunstmuseum. Falls möglich, können die Sketche auch in chronologischer Reihenfolge angebracht werden.

\paragraph{Heat map}
Nun wird darauf verzichtet, jeden Sketch einzeln zu erklären. Zunächst verschafft sich jede Person selbst einen Eindruck aller Sketche, ohne zu diskutieren, da die Zeichnungen sowieso selbsterklärend sein sollten. Dafür werden 20 bis 30 kleine Sticker an jeden Teilnehmer verteilt. Diese werden an besonders gute Teile der Sketche geklebt. Falls hier eine Idee besonders heraussticht, können dort auch 2-3 Aufkleber angebracht werden. Falls ein Sketch Fragen aufbringt, werden diese auf eine Haftnotiz unter die Zeichnung geklebt.

\paragraph{Speed Critique}
Diese Übung folgt wiederum einer klaren Struktur, weshalb Timeboxing durchaus empfohlen wird. Das Zeitlimit beträgt drei Minuten pro Sketch. Der Facilitator trägt einen Sketch vor und betont besonders herausstechende Teile. Auch die Teammitglieder dürfen besonders wichtige Details aufbringen, welche der Facilitator eventuell nicht aufgeschrieben hat. Ein Freiwilliger aus dem Team sollte während dieser Übung alle wichtigen Dinge auf Haftnotizen festhalten und über dem Sketch anbringen. Während der restlichen Zeit werden noch Fragen und Bedenken im Team geklärt. Bis zu diesem Zeitpunkt bleibt der Zeichner dieses Sketches still und fügt am Ende noch nicht aufgebrachte Details hinzu und beantwortet Fragen. Danach wird derselbe Prozess am nächsten Sketch angewandt.

\paragraph{Straw Poll}
Nun bekommt jeder Teilnehmer einen großen Sticker. Das Team wird dazu aufgerufen, das Ziel und die Sprint-Fragen nochmal für sich zu wiederholen. Innerhalb der nächsten zehn Minuten schreibt jede Person seine Stimme dafür auf, welche(r) Sketch(e) prototypisiert werden sollte. Diese Entscheidung kann ein gesamter Sketch oder auch nur ein kleiner Teil eines Sketches sein. Nach den zehn Minuten klebt jedes Teammitglied seinen Sticker auf einen der Sketche und erklärt seine Stimme kurz.

\paragraph{Supervote}
Die endgültige Entscheidung wird allerdings von den Decidern getroffen. Dafür bekommt jeder Decider drei große Sticker mit deren Initialen. Diese Sticker werden nun an den Sketches angebracht, welche am nächsten Tag in einen Prototyp verwandelt werden sollen. Hier ist wichtig, dass dabei nur die Decider-Stimmen zählen, die Stimmen des restlichen Teams unterstützen die Decider nur in ihrer finalen Entscheidung. Alle Sketches mit Supervotes werden nun nebeneinander an einer anderen Wand angebracht, alle anderen bleiben trotzdem kleben.

\paragraph{Rumble}
Nach der Entscheidungsrunde kann es nun passieren, dass mehr als ein Sketch überzeugt hat. Falls das nicht der Fall ist, kann diese Aufgabe übersprungen werden. Ansonsten wird in einer kurzen Gruppendiskussion entschieden, ob die Gewinner der Supervotes in dem Test an Tag 5 gegeneinander konkurrieren sollen oder ob es möglich ist, diese zu kombinieren. Der Vorteil einer Kombination ist, dass der Prototyp dann ausführlicher sein kann, da nur ein Prototyp erstellt werden muss. Wenn die Sketche aber sehr unterschiedlich sind, macht es auch Sinn, beide weiterzuentwickeln. Der große Vorteil daran ist, dass sich im Test klar herausstellen sollte, welches Produkt besser beim Kunden ankommt. 

\paragraph{Storyboard}
Die verbleibende Zeit von Tag drei wird genutzt, um ein Storyboard zu erstellen. Das besteht aus 10 bis 15 Rechtecken, in welche ein detaillierter Aublauf des Tests gezeichnet wird. Dafür wird eine Person aus der Gruppe als Zeichner an ein weiteres Whiteboard geholt, während das ganze Team in einer offenen Diskussion über die einzelnen Schritte und Übergänge entscheidet. Gestartet wird mit einer Eröffnungsszene. Diese soll den Kunden auf möglichst natürlichem Wege zum Produkt hinführen. Beispiele dafür sind der App Store für mobile Applikationen oder ein Zeitungsartikel. Alle weiteren Schritte hängen sehr vom Produkt an sich ab, deshalb gibt es dafür nur grobe Beispiele. Eine wichtige Regel ist, dass man verwenden soll, was man in den letzten Tagen erarbeitet hat und an dieser Stelle keine neuen Ideen mehr entwickelt. Außerdem sollte man sich im Storyboard nicht an der Wortwahl oder unwichtigen Details aufhängen, darüber kann man sich im Nachhinein immer noch Gedanken machen. Wie bereits die Tage zuvor trifft der Decider alle Entscheidungen, falls sich das Team uneinig ist. Am Ende sollte der Ablauf des Storyboards ungefähr 10 - 15 Minuten dauern, da der Kunde zusätzlich noch Zeit braucht, um zu überlegen und Fragen zu stellen bzw. zu beantworten.

Der Grund für dieses Storyboard, bevor der Prototyp erstellt wird, ist, dass viele Details im Vorfeld geklärt und eindeutig definiert werden. Es passiert oft, dass Kleinigkeiten vernachlässigt werden, wodurch viel vermeidbarer Klärungsbedarf bei der Entwicklung entsteht. Dabei müssen weitere Diskussionen geführt werden und das produktive Arbeiten am Prototypen muss unterbrochen werden. Mit dem Storyboard wird dies zu einem Großteil eliminiert.

\subsection*{Tag 4}
Dieser Tag wird vollständig dafür genutzt, den Prototypen zu erstellen. Da es sich um ein Modell handelt, reicht es vollkommen, nur eine Fassade des Produkts zu bauen. Ein komplettes Produkt zu entwickeln, nur um es mit potentiellen Kunden zu testen, ist aus verschiedenen Gründen nicht sinnvoll. Einer dieser Gründe ist der Zeitfaktor. Es dauert viel länger, das Produkt zu entwickeln obwohl man daraus keinen Mehrwert gewinnt. Der Testperson würde dieser Unterschied meist nicht auffallen. Außerdem sollte immer im Hinterkopf behalten werden, dass ein Prototyp eventuell nach negativen Resultaten aus den Tests komplett verworfen wird. Je länger sich eine Person mit der Entwicklung beschäftigt, desto höher ist der emotionale Wert, was wiederum kontraproduktiv sein kann. Nachdem das mit dem gesamten Team abgeklärt ist, fängt nun die eigentliche Arbeit am Prototypen an.

Zunächst muss festegelgt werden, welche Tools verwendet werden. Die Optionen dafür variieren je nach Produktart. Beispiele für Software sind PowerPoint, Keynote oder andere on-screen Prototyping Tools.

Um strukturierte Arbeit zu garantieren, werden einzelne Aufgaben durch den Facilitator an das Team verteilt. Die Aufgaben sind wie folgt definiert:
\begin{itemize}
	\item mindestens 2 Maker:
	Erstellen die einzelnen Bestandteile und den Ablauf des Prototyps.
	\item 1 Stitcher:
	Sammelt die einzelnen Bestandteile der Maker, fügt diese zusammen und sorgt für Einheitlichkeit.
	\item 1 Writer:
	Schreibt alle Texte, die in dem Muster vorkommen.
	\item mindestens 1 Asset Collector:
	Sammelt Bilder, Icons etc.
	\item 1 Interviewer:
	Schreibt ein Interview Skript und sollte nicht an der Entwicklung des Prototyps beteiligt sein.
\end{itemize}

Nach der Rollenverteilung werden Teile des Storyboards auf die einzelnen Personen aufgeteilt, die daran arbeiten sollen. Der Stitcher wird zuerst auch als Maker behandelt, bis genügend Material produziert wurde, um die Einzelteile zusammenzufügen. Nachdem der Prototyp fertig zusammengeschnitten wurde, wird ein Testlauf durchgeführt. Es sollte danach noch genug Zeit eingeplant werden, um eventuelle Fehler oder Probleme zu beseitigen.

\subsection*{Tag 5}
Der letzte Tag wird dafür genutzt, den Prototypen mit potentiellen Endkunden zu testen. Dafür werden zwei separate Räume benötigt, da der Test nur zwischen dem Interviewer und der Testperson stattfindet, während sich das restliche Team in einem anderen Raum aufhält. Die Testläufe werden mittels Video-Übertragung in den Raum projiziert, in dem das Sprint-Team sitzt. So kann jeder Teilnehmer Notizen machen ohne die Testperson zu stören oder zu verunsichern. Außerdem kann sich so der Interviewer voll und ganz auf den Test und die Person konzentrieren. Es sollten etwa fünf Testpersonen eingeladen werden, da man von dieser Anzahl an Personen am meisten lernt. So werden im Schnitt 85\% aller Fehler bereits von fünf Testpersonen entdeckt. Außerdem sollte das Team eine Fehlerquelle vermutlich genauer betrachten wenn mindestens zwei aus fünf Testern das gleiche Problem mit dem Prototypen haben.\cite{nielsen1993mathematical}
Darüber hinaus kann man so alle Interviews über einen Tag verteilt führen, sodass jede Sitzung eine Stunde dauert mit 30 Minuten Pause dazwischen. Der Ablauf der Interviews soll einem einfachen Muster folgen. Zunächst wird die Testperson freundlich Willkommen geheißen. Es muss außerdem vorab geklärt werden, ob die Person mit der Video-Aufzeichnung einverstanden ist. Dann sollte der Interviewer Fragen über die Testperson stellen, um seine Reaktionen oder Antworten besser einschätzen zu können. Damit kann auch gut auf das Produkt hingeleitet werden. Dieses wird dann vorgestellt und dem Kunden zur Verfügung gestellt. Durch eine offene und freundliche Atmosphäre soll der Tester so dazu angeregt werden, konstruktives und ehrliches Feedback zu geben. Nachdem der Gast einen kurzen Eindruck in das Produkt bekommen hat, bekommt er anschließend die Aufgaben, welche der Interviewer am Tag zuvor erstellt hat. Diese sollten nicht zu detailliert formuliert sein, der Tester sollte schließlich eigenständig mit dem Produkt interagieren. Abschließend werden noch allgemeine Fragen zu dem Produkt und dem Test gestellt.

Nach den Interviews versammelt sich das gesamte Team noch einmal zusammen in dem Raum mit den Notizen. Diese sollten auf Haftnotizen an einem Whiteboard kleben, unterschieden in die Kategorien positiv, negativ und neutral. Jeder Teilnehmer hat nun 5-10 Minuten Zeit, um sich die Notizen in Ruhe durchzulesen. Dabei soll jeder potentielle Muster entdecken und notieren. Danach wird über die erkannten kurz diskutiert. Zu diesem Zeitpunkt sollte der Decider in jedem Fall dazu in der Lage sein, eine Entscheidung darüber zu treffen, wie weiter vorgegangen wird. Damit ist der Sprint offiziell beendet.

\subsection*{Resultat}
Am Ende eines Sprints hat das Team innerhalb von fünf Tagen sehr viel über das Projekt gelernt. Dazu zählen auch negative Erfahrungen, die man sonst wahrscheinlich erst viel später im Projekt gemacht hätte. Auch wenn der Prototyp in jedem Test sehr schlecht ausfällt, ist diese Erkenntnis von großer Bedeutung. 

Ein weiterer großer Vorteil eines Sprints ist die große Nähe zu den Endkunden. Diese sind der Grund, warum das Projekt überhaupt durchgeführt wird und werden oft viel zu wenig in die Entwicklung eingebunden. Durch das Durchführen von mehreren Sprints kann es einem Team auch helfen, diese Kundennähe zu einer Gewohnheit zu machen.
\cite{Sprint}