\chapter{Schlusswort}
\section{Zusammenfassung}
Abschließend lässt sich sagen, dass die Kundenorientierung, welche alle zu validierenden Konzepte gemeinsam haben, durchaus sinnvoll ist. Das kann auch an oben beschriebenem Beispiel belegt werden. So hätte das Team vermutlich viel Zeit in die Umsetzung der Transaktion investiert, welche sich als unnötig herausgestellt hat. Durch die Lean Startup Methoden konnte dies im Vorhinein eliminiert werden.
%Im Rahmen der vorliegenden Arbeit wurden bereits teilweise bestehende Netzwerkkomponenten in einem Fahrzeugnetzwerk optimiert und mit weiter\add{er} Funktionalität erweitert. Ziel war es ein\add{e} Möglichkeit zu entwickel\change{t}{n}, einer Applikation die im Netzwerk gestartet werden sollte, verschiedene Applikationsparameter mit zu geben, anhand derer ein Fahrzeugnetzwerk automatisiert konfiguriert werden kann. Die Übergabe der Parameter soll mittels einem Applikationsmanifest geschehen, welches in Kapitel \ref{sec:Manifest} der Arbeit eigens spezifiziert wurde. Anhand der Eigenschaften einer Anwendung wurde die zentrale Stelle des Netzwerk, der Netzwerk-Manager, funktionell so erweitert, dass er selbstständig Netzwerkkonfigurationen erstellen kann. 
%
%Die Erstellung einer Konfiguration soll mittels verschiedener Features-Erweiterungen umgesetzt werden, welche in Kapitel \ref{sec:Config} erarbeitet wurden. So kann der Manager zukünftig beispielsweise automatisiert neu Pfade für die Kommunikation zweier Netzwerkteilnehmer festlegen. Durch die Übergabe der Anforderungen einer Applikation in Form eines Manifest\add{s}, weiß der Netzwerk-Manager\add{,} welche Konfigurationen für ein Netzwerk nötig sind und setzt diese mittels der Features um. Funktionen, wie die Berechnung der Datenrate und Echtzeitfähigkeit, sind unerlässlich für das neu\add{e} Netzwerk, da nur so ein fehlerfreier Ablauf des Systems gewährleistet werden kann.
%
%%\newpage
%
%In Kapitel \ref{sec:Impl} wurden die bisher erarbeiteten Punkte erstmals softwaretechnisch umgesetzt. Aufgrund des definierten Umfang\add{s} der Abschlussarbeit wurde sich bei der Implementation auf einen Beispielfalles bezogen. Die Auswahl des Testfalles wurde unter Beachtung der technischen Umsetzbarkeit getroffen. Leider können zum jetzigen Zeitpunkt noch nicht alle im Projekt verwendeten Endgeräte die geforderten Technologien aus der Arbeit umsetzen. Aus diesem Grund wurde sich auf Ingress Filtering beschränkt, da dies bereits \change{F}{f}unktionell umsetzbar ist. 
%\newpage
%In der Arbeit waren softwareseitige Unit-Tests zur Gänze ausreichend, da durch die Features eine korrekte Liste an Befehlen erstellt wurde, die einen Netzwerkswitch konfigurieren können. 

\section{Fazit und Ausblick}
%Das Erstellen eines dynamischen Fahrzeugsnetzes für zukünftige E/E\change{ }{-}Architekturen birgt große Herausfordungen und bedingt das Anwenden von bisher unbekannten Methoden. Diese Arbeit bildet einen Schritt hin zu einem Fahrzeugnetzwerk mit eine\change{r}{m} zentralem Manager, der remote die Netzwerkeinstellungen aller Netzwerkteilnehmer verwaltet und überwacht. Anhand der erarbeiteten Konzepte, wie QoS oder TSN, konnte ein Manifest erstellt werden, dass die wichtigsten Parameter für eine Applikation im Bezug auf ein Netzwerk enthält. Der Prozess der Parameterfindung wird in einem Netzwerkmanager für ein zukünftiges Fahrzeugnetzwerk eine zentrale Rolle einnehmen und stellt ein hochgradig nicht-triviales Problem dar, da dafür ein hohe\change{s}{r} Grad an Systemwissen notwendig ist und Konfigurationsparameter sinnvoll definiert werden müssen. Die kommenden Schritt\add{e} werden die in Abhandlung erarbeiteten Erkenntnisse im Rahmen des A3F\change{ }{-}Forschungsprojekt\add{s} softwareseitig umzusetzen. Parallel dazu wird in einer weiteren Forschungsarbeit ein Konzept zu Sicherstellung auf Korrektheit der Konfigurationsfindung begonnen. Dies wird notwendig sein, um eine Zulassung der Technologie nach den aktuellen Gesetzen zu ermögliche\add{n}. Die Forschungsarbeit wird jedoch nicht der einzige Prozess sein, der hierzu nötig sein wird, sonder\add{n} ist ein weiterer Schritt zur Etablierung der Technologie in neuen Fahrzeugnetzwerken.