\section{Motivation der Arbeit}
Es ist statistisch erwiesen, dass rund 70\% der Startups im IT-Bereich scheitern.  \cite{CBInsights_failure} Dabei gehen Experten, wie beispielsweise \citeA{marmer2011startup} davon aus, dass insgesamt mehr als 90\% aller Gründungen nicht erfolgreich sind. Eine der wichtigsten Ursachen dafür ist beispielsweise die fehlende Nachfrage, welche bei 42\% der gescheiterten Gründungen genannt wird. Außerdem ist bei ungefähr 17\% die schlechte Benutzerfreundlichkeit des Produktes ein Problem. Weitere Schwierigkeiten stellen etwa das falsche Team, fehlendes Budget und Konkurrenzkampf dar. \cite{CBInsights_reasons} 

Diese Statistiken stehen dem Scheinbild entgegen, dass aus jeder simplen Idee ein Unternehmen entstehen kann. Doch die Idee alleine genügt nicht. Es ist unter Anderem auch wichtig, diese auf die Zielgruppe abzustimmen und richtig zu vermarkten. 
\begin{quote}
,,I realized, essentially, that we had no customers because no one was really interested in the model we were pitching. Doctors want more patients, not an efficient office.'' \cite{CBInsights_reasons}
\end{quote}
Das Zitat eines gescheiterten Unternehmers zeigt außerdem beispielhaft, wie wichtig es ist, die Endkunden in die Entwicklung miteinzubeziehen, damit deren Wünsche berücksichtigt werden können. Ein weiterer Erfahrungsbericht von \citeA{preventFailure} beschreibt, welche Maßnahmen ein Scheitern eines Startups im Vorhinein verhindern könnten. Dabei nennt er zuerst das Fokussieren auf die Zielgruppe, anstatt rein auf die Implementierung. Denn es ist ein verbreiteter Fehler, dass bereits mit der Entwicklung begonnen wird, bevor die Rahmenbedingungen genau festgelegt wurden. Die Entwickler denken oft, ein klares Bild vom Endprodukt zu haben, ohne es mit dem Kunden oder der potentiellen Zielgruppe abzustimmen. Daher wird das Produkt aus Entwicklersicht umgesetzt, was in fehlender Nachfrage resultieren kann, da es meist nicht dem entspricht, was vom Endnutzer gebraucht wird.

Um den oben beschriebenen typischen Startup-Schwierigkeiten entgegenzuwirken, werden Prozesse entwickelt, mit welchen jede Idee vermarktbar und erfolgreich zu sein scheint. Die Erfinder Lean-Konzepte versprechen großen Erfolg und in immer mehr Unternehmen werden Innovationsprozesse angewandt. Diese bauen alle auf einem gleichen Prinzip auf: Das Produkt perfekt auf den Kunden abzustimmen, indem die Nutzer bereits in die Entwicklung miteingebunden werden. Ein Startup kann nur dann erfolgreich sein, wenn ein ausreichend großer Markt für das Produkt existiert. So wird dies auch als der wichtigste Aspekt erfolgreicher Unternehmensgründungen genannt. Daher ist anzunehmen, dass Lean Prozesse, welche in Kapitel \ref{sec:Grundlagen} genauer erklärt sind, die Anzahl der scheiternden Startups reduziert. Das entspricht allerdings nicht der Realität, weshalb Lean-Konzepte im Zuge dieser Arbeit kritisch betrachtet werden.\cite{patel201590} Diese werden von einem Beispielstartup namens Agrishare getestet, um darauf aufbauend Rückschlüsse über den Erfolg der Konzepte zu ziehen. In dieser Arbeit wird unter Anderem die Umsetzung der Konzepte durch das Agrishare-Team beschrieben. Außerdem fließen die Meinungen und Ergebnisse von Agrishare in die Bewertung der Prozesse mit ein.

Das Team, bestehend aus Studenten der OTH Regensburg, formte sich während des Hackathons \textit{Hackaburg 2018}, einer Veranstalltung in Zusammenarbeit mit der OTH Regensburg. Das ist ein Event, an welchem über einen festen Zeitraum verschiedene Software- oder Hardwareprobleme gelöst werden. Die Projekte werden teilweise von Firmen gestellt. Zusätzlich ist es möglich, eine eigene Idee umzusetzen und vorzustellen. Die besten Ausarbeitungen werden am Ende mit einem Preis belohnt, wie die Android-App des oben genannten Teams. Agrishare kann die Jury in zwei Runden überzeugen und gewinnt so den ersten Preis des Hackaburgs - wie die Veranstaltung in Regensburg offiziell heißt. Aufgrund des positiven Feedbacks und dem hohen Engagement im Team entscheiden die Teilnehmer, die Idee weiter zu verfolgen. Daraus entsteht der Beschluss, ein Startup um das vielversprechende Projekt zu gründen.

\newpage

%Es stellt sich also die Frage, ob potentiell vermeidbare Probleme bei der Startup-Gründung mithilfe von modernen und erfolgsversprechenden Prozessen eliminiert werden können. Um dies zu beantworten wird die Prozesskette ,,The Innovator's Method'', welche die berühmtesten modernen Methoden enthält, mit dem oben beschriebenen Startup-Team durchlaufen. Anhand dieses Beispiels sollen die Prozesse evaluiert werden und diese Ergebnisse am Ende mit anderen Erfahrungsberichten verglichen. 