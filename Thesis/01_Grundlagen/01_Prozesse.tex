\section{Sprint von Google Ventures}
\section{The Lean Startup}

The Lean Startup beschreibt eine Methode, erfolgreich ein Startup zu leiten, angelehnt an Lean Manufacturing. Dies besteht aus unterschiedlichen Methoden, welche häufigen Fehlern im Umgang mit Startups entgegenwirken sollen, um diese Firmen vor dem Scheitern zu bewahren.

Eric Ries startet hier mit einer eigenen Definition von einem Startup, um es von anderen Unternehmensformen klar abzugrenzen. Diese lautet: \textit{,,a human institution designed to create new products and services under conditions of extreme uncertainty.''}
Das Schlüsselwort \textit{uncertainty} (zu deutsch: Unsicherheit) in diesem Satz ist hierbei essentiell, da die Hauptschwierigkeit eines Startups darin besteht, ein Produkt zu entwickeln, welches tatsächlich gebraucht wird. Das kann im Vorhinein nur bis zu einem gewissen Grad getestet werden, weshalb diese extreme Unsicherheit eine große Hürde darstellt.

Diese fünf Prinzipien formen den Grundgedanken der Lean Startup-Methode mit:
\begin{itemize}
	\item \textbf{Erfinder gibt es überall}
	
	Ein neues Produkt kann in jeglichen Umgebungen entstehen. Daher ist diese Methode nicht auf eine Branche oder Unternehmensgröße zugenschnitten. Startups können also auch in großen Firmen entstehen und können aus einer Prozess-Perpektive trotzdem mit einem Garagen-Projekt verglichen werden.
	\item \textbf{Gründung ist Management}
	
	Nicht selten wird komplett auf Prozessplanung verzichtet, sobald etablierte Methoden fehlschlagen. Dies sieht Ries allerdings als großen Fehler, da der Erfolg eines Startups oft von dem Entstehungsprozess abhängig ist.
	\item \textbf{Validated learning}
	
	Ein Startup muss zuerst lernen, wie ein erfolgreiches Unternehmen um das Produkt errichtet werden kann. Das kann durch regelmäßige Experimente herausgefunden werden, wodurch wichtige Erkenntnisse gewonnen werden.
	\item \textbf{Bauen - Messen - Lernen}
	
	Das beschreibt Iterationen aus dem Bau eines neuen/erweiterten Produkts, der Messung der Kundenreaktionen und dem Lernen ob diese Strategie erfolgreich ist. Diese Abfolge soll den Arbeitsalltag von Startups beschreiben.
	\item \textbf{Innovation accounting}
	
	Ein Startup muss entscheiden, wie Erfolge gemessen werden können, wie Meilensteine gesetzt werden können und wie priorisiert werden soll. All diese Dinge basieren in etablierten Unternehmen auf Budget- und Zeitplanung. Das ist allerdings in dem Sonderfall Startup keine gute Herangehensweise, wie im Folgenden erläutert wird.
\end{itemize}

\paragraph{Vision}
Die Vision liegt jeder Idee zugrunde. Sie beschreibt das Ziel, welches durch das Produkt auf lange Sicht erreicht werden soll. Darauf aufbauen entwickelt sich eine Strategie nach welcher das Produkt entwickelt werden soll. Diese basiert wiederum als Fundament für das eigentliche Produkt. Eine der größten Herausforderungen für ein Startup ist es, diese drei Bausteine so anzupassen, dass ein erfolgreiches Unternehmen daraus entstehen kann. Dabei muss das Produkt wahrscheinlich relativ oft angepasst werden, was Optimisierung genannt wird. Außerdem kann es durchaus passieren, dass die Strategie abgeändert wird, was als Pivot beschrieben wird. Die Vision selbst wird äußerst selten geändert, da sie der Grundbaustein des gesamten Startups ist.