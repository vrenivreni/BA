\section{The Lean Startup}
\subsection*{Validated Learning}
Bereits zu Beginn des Projektes wird bei Agrishare darauf geachtet, jegliche Hypothesen über das Nutzungsverhalten zu überprüfen. 

So steht beispielsweise die These im Raum, dass Landwirte aktuell ihre Arbeiten über einen Großkonzern vermitteln, dann aber eigenständig abrechnen, um sich die Zuschläge für diesen Konzern zu sparen. Überraschenderweise stellt sich heraus, dass genau das Gegenteil der Fall ist. Meist kennen sich Landwirte ohnehin untereinander und wissen, welcher Bauer in der Nähe welche Maschinen besitzt und welche Arbeiten verrichten kann. Das heißt, die Vermittlung geschieht eigenständig. Jedoch erfolgt die Abrechnung ausschließlich über oben genannten Konzern, da dieser für einwandfreie Rechnungsstellung und Dieselzertifikate sorgt. Das wiederum verringert den bürokratischen Aufwand für die Landwirte.

Außerdem wird das Konzept, Maschinen ausschließlich versichert zu vermieten, anfänglich als sinnvoll angesehen. Nach einigen Gesprächen mit potentiellen Endkunden wird dies jedoch verworfen. Der Grund dafür ist, dass Landwirte ohnehin eigenständig sind und daher über ausreichend Versicherungsschutz verfügen müssen. Zusätzliche Absicherung würde daher keinen Wert für den Endnutzer bringen.

\subsection*{Innovation Accounting}
Nach dem Innovation Accounting Konzept wird zuerst ein MVP entwickelt, um die Baseline aufzustellen. Das MVP entsteht aus dem zuvor durchgeführten Sprint Prozess.

In der ersten Build-Measure-Learn Iteration wird auf der Startseite ein Abschnitt eingefügt, der dem Nutzer personalisierte Vorschläge zeigt. Angelehnt an persönliche Interessen, das frühere Nutzungsverhalten und den aktuellen Standort des Kunden werden attraktive Angebote für Maschinen oder Dienstleistungen angezeigt. Der Nutzertest hat hier eindeutig gezeigt, dass das Feature wertvoll ist. 

\subsection*{Pivot}
Probleme entstehen in der nächsten Iteration. Da die Plattform den kompletten Vermittlungs- und Transaktionsprozess erleichtern soll, ist der Transaktionsteil kritisch. Dabei muss darauf geachtet werden, dass der Geldfluss problemlos vonstatten geht und auch die Rechnungsstellung einwandfrei funktioniert. Nachdem das Team viel Zeit in die Detailplanung der Transaktionen investiert hat, fällt im Nutzertest auf, dass die Transaktion bereits digitalisiert ist. Darüber hinaus hat die Zielgruppe kein Bedürfnis danach, diesen Prozess abzuändern, denn das aktuell genutzte Produkt funktioniert einwandfrei. Allerdings ist die Vermittlung von Maschinen und Dienstleistungen nach wie vor nicht digitalisiert. Nach diesen neuen Erkenntnissen entscheidet das Agrishare Team, ein Pivot durchzuführen. Dieses Pivot kann am besten durch ein Zoom-In Pivot beschrieben werden. Denn das Produkt wird auf die Vermittlung eingeschränkt, was ursprünglich nur einen Teil des Gesamtproduktes dargestellt hat. Diese Entscheidung stimmt das Team sehr positiv, da dadurch eine große Verantwortung und damit eines der größten Risiken wegfällt. Außerdem stellt das Produkt nun keine direkte Konkurrenz für oben genannten Großkonzern dar, welcher sich ohnehin auf die Transaktion spezialisiert.

\subsection*{Batches}
Das Batch-Konzept wird im Agrishare Team ebenfalls angewandt. So wird das Backend on-demand entwickelt und in enger Zusammenarbeit mit den Fortschritten im Frontend angepasst und optimiert. Im Gegensatz dazu wäre es auch möglich gewesen, Frontend und Backend separat voneinander zu entwickeln und im Nachhinein zusammenzufügen. Wie Ries jedoch voraussagt, würden dadurch große Inkompatibilitäten entstehen. Dadurch müsste sehr viel zusätzliche Arbeit investiert werden, die beiden fertigen Blocks aufeinander abzustimmen. Durch kleinere Blöcke, welche regelmäßig zusammengeführt und aufeinander abgestimmt werden, wird dieser Arbeitsaufwand minimiert.

\subsection*{Engine of Growth}
Das Wachstum des Unternehmens soll mit dem \textit{viral engine of growth} gemessen werden. Diese Methode erweist sich hier als sinnvoll, da eine Plattform davon lebt, viele Nutzer in enger räumlicher Dichte zu haben. Außerdem macht es Sinn, die bereits vorhandene gute Vernetzung der Landwirte auszunutzen. Es wird angenommen, dass ein Landwirt, welcher die Plattform nutzt, vielen seiner Nachbarlandwirten davon berichten wird. Dies ist allerdings eine Prognose, die so nicht im Vorhinein überprüfbar ist. Daher kann die Unsicherheit, die Ries dazu treibt, das Lean Startup Konzept zu entwickeln, nie ganz aus dem Weg geräumt werden.

\subsection*{Evaluierung}
Nach den Aussagen des Agrishare Teams ist das Konzept des Lean Startups sehr sinnvoll. Allerdings besteht es aus Methoden, die logisch scheinen und nicht überraschend sind. Das heißt, es wird ohnehin angenommen, dass jedes Startup, das sich ansatzweise mit Herangehensweisen an Startup-Gründung auseinandersetzt, diese oder ähnliche Methoden aus gesundem Menschenverstand heraus anwenden. Objektiv betrachtet, ist es einfach, solche Aussagen zu treffen, nachdem sich bereits mit dem Prozess auseinandergesetzt wurde. Das Team kann daher keine Aussagen darüber treffen, wie diese Dinge angegangen worden wären, wäre Lean Startup nicht bekannt gewesen. Anhand der Beispiele, welche Ries in seinem Buch beschreibt, wird dies Dinge in der Praxis selten angewandt. Allein durch diese groben Fehler, kam er letztendlich auf die Idee, Lean Startup zu entwickeln.


