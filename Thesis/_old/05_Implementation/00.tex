\chapter{Implementierung der Features}\label{sec:Impl}
Das Ergebnis aus der dem Entwurf des Maifestes wird eine erste Implementation eines Testszenarios sein. Dies ist ebenfalls Bestandteil dieser Arbeit und wird mittels einiger der in Kapitel \ref{sec:FeatureDef} definierten Features umgesetzt. In dieser Arbeit werden nicht alle genanten Features umgesetzt, sondern lediglich ein Szenario im System, welches bereits softwareseitig umsetzbar und testbar ist. Die benötigten Klassen hierzu werde weitestgehend neu implementiert, lediglich Schnittstellen sind bereits im System enthalten und werden als gegeben betrachtet. 

Einen generellen Überblick aller Relevanten Komponenten und deren Kommunikationsschnittstellen, sog. Interfaces, werden in Folgenden Komponentendiagramm kurz dargestellt. Dies soll die Verbindungen zwischen den Komponenten, sowie die Vernetzung innerhalb der Unterkomponenten vereinfacht darstellen.
\singlefigureMax{\label{fig:KomponentenSystem}}{99_Dias/ComponentManager.png}{Komponentendiagramm Netzwerkmanager}

\newpage

Da besonders im Bereich Autonomes Fahren ein hoher Anspruch an die Ausfallsicherheit für eine Anwendungen im System gilt, werden im Nachgang noch mehrere Unit-Tests durchgeführt. Anhand dieser kann die Software auf verschiedene Eingabefälle geprüft werden. In Unit-Test werden isoliertet Softwareeinheiten, wie einzelne Klassen, mittels definierter Eingabe geprüft, um eine zuvor festgelegte Ausgabe zu erzeugen. Nur wenn dieses Ergebnis mit der gewünschten Ausgabe übereinstimmt, wird der Test als bestanden gewertet. Dies garantiert eine möglichst vielschichtige Abdeckung an Testfällen. Unit-Test geben aber keine Auskunft über das Zusammenspiel von Klassen über Schnittstellen, dies muss im Nachgang noch manuell getestet werden.

\section{Testfall Ingress Filtering}\label{sec:Ingress}
Der gewählte Testfall bezieht sich auf das Ingress-Filtering eines Datenstroms, welches bereits in Abschnitt \ref{sec:Filter} kurz angesprochen wurde. Dieser Filter wird empfängerseitig an den jeweiligen Ports, der sich in der Kommunikationsstrecke befindenden Geräten, konfiguriert. Es handelt sich dabei um eine Filterung auf  Layer 2 Eben. Ingress-Filterring ist ähnlich aufgebaut wie eine Firewall. Im Initialzustand sind alle Kommunikationen verboten. Erst wenn Regeln definiert werden, die das ein- oder ausgehen von Nachrichten erlauben, kann eine Kommunikationsstrecke aufgebaut werden. So werden bei einer Ende-zu-Ende Kommunikation lediglich die beiden Endpunkte zugelassen, die auch in den Regeln festgelegt sind. Hierbei werden die Partner anhand ihrer \ac{MAC}-Adressen in den Regeln definiert. So wird sichergestellt, dass nur die gewünschten Teilnehmer die Datenpakete empfangen und versenden können. 

Für die Regelung der Kommunikation haben sich hauptsächlich zwei Verfahren etabliert. 
\begin{itemize}
	\item \emph{Application Whitelisting}
	\newline Dieses Verfahren verbietet zuerst jegliche Kommunikation. Es müssen explizite Regeln für den Nachrichtenaustausch angelegt werden, um diese zu erlauben. In einem Netzwerk mit vielen Teilnehmern, bei denen nur wenige untereinander eine Verbindung aufbauen dürfen, ist dies ein vielversprechender Ansatz. Es werden wenige Regel benötigt, da Verbote nicht definiert werden müssen.
		
	\item \emph{Application Blacklisting}
	\newline Beim \emph{Blacklisting} ist grundsätzlich jegliche Verbindung zugelassen. Ein Verbot der Kommunikation zweier Partner kann nur durch eine Regel definiert werden. Dieser Ansatz macht nur Sinn, wenn es sehr viel Netzwerkteilnehmer gibt, bei denen die meisten untereinander Kommunizieren dürfen. Es müssen somit nur wenige Regeln für Verbote definiert werden.
\end{itemize}

\newpage

Für dies Arbeit wird erste Ansatz verwendet, da es eine überschaubare Anzahl an Kommunikationspartnern gibt, die wiederum aus Sicherheitsgründen nicht alle untereinander kommunizieren dürfen. Ein weiter Vorteil des Ansatzes ist, dass durch das explizite Erlauben der Kommunikation das Fehlerrisiko einer Verfälschung der Daten durch andere Teilnehmer minimiert wird. Dies kann durch einen anderen Netzwerkteilnehmer erfolgen, der die Nachricht auch empfangen konnte und sie abgefälscht weiterleitet. Ein Verfälschen der Nachrichten kann sowohl mutwillig durch eine manipulierten Netzwerkteilnehmer, als auch unwissentlich durch einen Defekt erfolgen. Beide Fälle müssen im System abgefangen werden, da durch sie ein Risiko entsteht.

\section{Konfiguration der Endgeräte}\label{sec:ConfigEndgerät}
In der Umsetzung der Arbeit wird sich auf die Konfiguration der Switche in einer Kommunikationsstrecke beschränkt, weshalb in diesem Fall ein Endgerät gleichzusetzen ist mit einem Switch. Generell gibt es zwei Möglichkeiten der Konfiguration der Endgeräte. Die erste ist eine Erstellung einer Konfigurationsdatei im \ac{XML} Format, welche alle generierten Einstellungen für den jeweiligen Switch enthält. Durch aufspielen dieser Datei auf ein Endgerät, kann dieser neu konfiguriert werden. Die zweite Möglichkeit, auf die sich auch dies Arbeit bezieht, ist das Nutzen eine im Projekt eigens für die Entwicklung geschriebene Library, welche alle Funktionen zur Konfiguration eines Endgerätes enthält. Diese wurde bereits im Vorfeld entwickelt und soll nun im Netzwerkmanager integriert und gesteuert werden. Hierzu wird von den in Abschnitt \ref{sec:FeatureDef} definierten Featuren, eine Liste an Instruktionen erzeugt, welche am Ende einer Konfigurationsfindung von der Funktionalität der Library abgearbeitet werden soll.

\section{Umsetzung der Parametereinlesung}
Wie in Kapitel \ref{sec:Config} bereits erwähnt, bekommt der Netzwerkmanager seine Parameter mittels \ac{JSON} File übermittelt. Dies muss im ersten Schritt ausgelesen werden, und den entsprechenden Variablen im System zugeordnet werden, um im Nachgang damit zu arbeiten. Hierzu wurden die Klassen \emph{json\_translator} und \emph{in\_car\_application} implementiert. Diese ermöglicht die Konvertierung der \ac{JSON}-Parameter in die benötigten Variablentypen, welche in der Klasse \emph{in\_car\_application} definiert sind. Der \emph{json\_translator} ist auch in der Lage zu erkennen ob eine Anwendung mehrere Nachrichten versenden will und berechnet die Gesamtbandbreite der Applikation. Dies ist wichtig, damit der Netzwerkmanager später entscheiden kann, ob noch genügend Bandbreite für die Applikation vorhanden ist. 
\newpage
In der Klasse \emph{http\_connector} ist die in Abschnitt \ref{sec:REST} erwähne Technologie zum hinzufügen einer neuen Applikation implementiert, wird die Methode zum  \emph{adden} (dt. hinzufügen) einer neuen Anwendung zum System aufgerufen, starte diese das Einlesen der Parameter über den \emph{json\_translator} und gibt als Rückgabewert ein  Objekt vom Typ \emph{InCarApplication} zurück. In dem Objekt sind alle im Applikationsmanifest definierten Parameter abgelegt und können im sog. Netzwerkstatus (Klasse \emph{netwerk\_flow}) jederzeit von andern Klassen abgefragt werden. Für den weiteren Verlauf der Arbeit kann davon ausgegangen werden, dass eine Applikation vollständig in einem Objekt der Klasse \emph{in\_car\_application} abgebildet ist. 

\singlefigureMax{\label{fig:ClassJson}}{99_Dias/ClassJsonParam.png}{Klassendiagramm Parametereinlesung}

Der bereits erwähnte Netzwerkstatus spiegelt den aktuelle Stand des Netzwerkes wieder. So sind dort beispielsweise Datenflüsse, Laufende Anwendungen und  einige weitere Parameter enthalten. Der Status kann somit jederzeit von verschiedensten Funktionen abgefragt werden. Der Zusammenhang zwischen den verschieden Klassen die für die Parametereinlesung verantwortlich sind, ist im Klassendiagramm Abb. \ref{fig:ClassJson} dargestellt. Aufrufe von außen, wie beispielsweise durch den Orchestrator, erfolgen über den \emph{http\_connector}.



\section{Umsetzung der Parameterauswertung mittels Feature}\label{sec:Auswertung}
Nachdem die Parameter in den entsprechenden Variablen hinterlegt wurden, beginnt der Netzwerkmanager nacheinander die Features aus Abschnitt \ref{sec:FeatureDef} abzuarbeiten. Diese liefern die gewünschten Konfigurationsbefehle für den Switch in Form einer Befehlsliste zurück. Im Testfall Ingress Filtering wird mit dem Feature zum Aufbau der Kommunikationsstrecke begonnen. Dazu werden die vom Orchestrator übergeben \ac{ECU}'s als Start- und Endknoten verwendet. Die Kommunikationsstrecke über die Daten der Applikation übermittelt werden, wird als Flow oder Datenflow bezeichnet. In einem Flow sind die sendende und empfangende Applikation als Anfang- und Endpunkt abgespeichert, anhand dieser kann später eine zuordnen zwischen Applikation, Kommunikationspartnern und Datenflow erfolgen. Mithilfe der Klasse \emph{simple\_path\_resolution} werden mögliche Pfade zwischen den Endpunkten gesucht. Ist ein geeigneter Pfad für einen Flow gefunden, werden beide in einer Map mit \emph{Key-Value Pairs} abgelegt. So kann ein Pfad eindeutig einer Datenflow zugeordnet werden. 

Ein Pfad kann aus mehren Edges bestehen, welche wiederum eine Teilstrecke eins Pfades darstellt. Eine Edge entspricht einem Hop laut Netzwerkdefinition, d. h. dem Weg zwischen zwei Netzwerkgeräten. Edge sind notwendig, da ein Pfad nur Anfang- und Endgerät kennt, jedoch nicht alle Teilnehmer die ein Datenpaket beim Versenden passieren muss. Eine Edge hat ebenfalls eine Start- und Endkonten, diese werden im System als Nodes bezeichnet.

Nodes sind nichts anders als Geräte im Netzwerk wie beispielsweise ein Switch. In einem Node besitzen Attribute wie \ac{MAC}-Adresse, Ports und einem \emph{Key-Value Pair}, welches eine Verbindung zwischen Ports und Node herstellt. Dieses Mapping ist notwendig um später, in der Verarbeitung der Features, alle dem Pfad zugehörigen Nodes den entsprechenden Ein- und Ausgangsport zuordnen zu können, da ein Node in der Regel nicht nur einen, sondern mehrere Ports besitzt. 

Um die Zusammenhänge der einzelnen Klassen zu verdeutlichen, wurde in Abb. \ref{fig:network} ein Klassenmodel erstellt. Diese enthält in den einzelnen Komponenten lediglich die für die Arbeit relevanten Variablen und Funktionen.

\singlefigurePlus{\label{fig:network}}{99_Dias/ClassNetwork.png}{Klassendiagramm Netzwerkübersicht}

\newpage

Sobald die Pfadfindung abgeschlossen ist, werden aus der Klasse \newline\emph{feature\_based\_configuration\_resolution} heraus, nacheinander die einzelnen Features aus Kapitel \ref{sec:FeatureDef} aufgerufen. So wird das Feature Kommunikation, welches Ingress Filtering beinhaltet und in der Klasse \newline \emph{feature\_communication\_ingress\_filtering} implementiert ist, ebenfalls hier durchlaufen. Dieses sucht, mittels der übergebenen Applikation, dem Flow-Pfad Mapping und dem aktuellen Netzwerkstatus den zugehörigen Pfad für die Anwendung. Ist dieser gefunden, wird über alle Edges des Pfades iteriert und für alle Nodes auf Empfängerseite die entsprechenden Portregelungen für das Ingress Filtering in der in Abschnitt \ref{sec:ConfigEndgerät} erwähnten Liste mit Instruktion abgelegt. Sobald alle Edges durchlaufen sind, gilt das Feature als abgearbeitet und es kann mit dem nächsten Feature begonnen werden. Sind alle abgearbeitet wird die Instruktion Liste an die Schnittstelle zur Konfiguration der Endgeräten übergeben. Dies konfiguriert mit den entsprechenden Funktionen, aus der erwähnten Library, das Endgerät. Im Fall Ingress Filtering werden die Portregelungen, wie Filtereigenschaften für Datenpakete, mit den entsprechenden Ein- und Ausgangsadressen der Netzwerkgeräte, übermittelt.


%\singlefigureMax{\label{fig:ClassSwitch}}{99_Dias/SwitchConfig.png}{Klassendiagramm Switch-Konfiguration}

\subsection{Testing der Implementation Ingress Filtering}
Um sicher zu stellen das das Konzept der Features, zur automatischen Konfiguration, richtig arbeitet, müssen alle Teilelemente der Implementation mittel Unittests überprüft werden. Hierzu wurde zuerst eine Testklasse zu Parameter-Einlesung geschrieben, welche der Klasse \emph{json\_translator} ein Manifest im \ac{JSON} Format übergibt und die zurückgegebenen Ist-Werte mit den zuvor definierten Soll-Werten vergleicht. Dies soll sicherstellen das alle Parameter aus den Manifest in den richtigen Variablen des Netzwerkmanager landen. 

Die zweit Testklasse ist speziell für den Testfall Ingress Filtering implementiert. Es besitzt ein zuvor definierte Netzwerktopologie mit drei Netzwerkknoten. Anhand dieser wird der in Abschnitt \ref{sec:Auswertung} Prozess durchlaufen. Nach der Abbildung der Topologie in der Software, mittels Flows, Pfaden und Edges, wird die automatisierte Portregelungs-Konfiguration in der Klasse \emph{feature\_communication\_ingress\_filtering} angestoßen. Nach dem vollständigen Durchlauf des Features, wird die erstellte Instruktionsliste mit der erwarteten Ergebnis verglichen.

Beide Test konnten positiv bewertet werden, was sicherstellt das, das Konzept der automatischen Netzwerkkonfiguration auf diese Weise umsetzbar ist. Ein nächster Schritt ist es die Konfiguration auf eine realen Switch weiter zu leiten, um auch Hardwareseitig noch einige Tests durchführen zu können. Dies ist jedoch nicht mehr Teil dieser Arbeit, wird aber bereits im Projekt A3F weiter verfolgt.