\section{The Innovator's Method}
Die Schritte Insight und Problem werden am Beispiel Agrishare teilweise bereits vor dem Hackathon abgearbeitet. Dafür werden nicht die empfohlenen Prozesse angewandt, da zu diesem Zeitpunkt das Thema Startup-Gründung noch nicht im Raum steht.

So erkennt ein Teammitglied vor dem Event, dass es in diesem Bereich hohes Potential für Verbesserungen im Rahmen der Digitalisierung gibt. Aufgrund des landwirtschaftlichen Hintergrundes der Person, ist es dem Mitglied möglich, eigene Einblicke mitzubringen, da er selbst zur Zielgruppe gehört. Aufbauend auf diese Erkenntnis entwickelt der Teilnehmer eine grobe Idee, mit welcher er zu dem Hackathon kommt um ein Team für die Umsetzung dieser zu finden. Nachdem sich das Team gebildet hat, wird in der Gruppe über das Konzept diskutiert. Im Rahmen dieses Gespräches werden für das Problem einzelne Lösungsansätze entwickelt. Einer dieser Ansätze wird anschließend prototypisch implementiert. Zu diesem Zeitpunkt umfasst der Prototyp eine Android-basierte App. Während des Hackathon kann das Produkt nicht mit potentiellen Endkunden getestet werden, allerdings führt das Feedback der Jury, sowie die Ansichten des Teams selbst dazu, dass die Plattform eine Web-Anwendung werden soll, im Gegensatz zur App. Bereits zu diesem Zeitpunkt wird ein Konzept angewandt, welches in Kapitel \ref{sec:LeanStartup_Pivot} unter Pivot beschrieben ist, dem Team an dieser Stelle allerdings noch nicht bekannt ist. Unwissentlich entschließt das Gründerteam hier, ein Plattform Pivot durchzuführen. Der wesentliche Grund dafür ist, dass eine Web-Anwendung auf allen Endgeräten aufgerufen werden kann, unabhängig von der Größe und Art des Gerätes, sowie vom Betriebssystem. Daher muss nur eine Anwendung implementiert werden, im Gegensatz zu einer Realisierung der App, welche für jedes mobile Betriebssystem umgesetzt hätte werden müssen.

Nach dieser Konzeptänderung beschließt das Agrishare-Team einen Sprint durchzuführen, welcher in Kapitel \ref{sec:Sprint-Grundlagen} detailliert erklärt ist. Wie bereits beschrieben, wird im Rahmen dieses Workshops das Problem innerhalb des Gesamtprojektes eingegrenzt. Daher fällt der Sprint unter Anderem unter die Kategorie ,,Problem'' der Innovator's Method. Nichtsdestotrotz wird dabei ebenfalls ein Prototyp entwickelt und mit der Zielgruppe getestet. Was darauf schließen lässt, dass auch die Kategorie ,,Solution'' angeschnitten wird.