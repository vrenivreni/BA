\section{Evaluierung am Beispiel}
Wie bereits oben angedeutet, hatte das Projektteam allgemein sehr niedrige Erwartungen an den Prozess. Es wurde kein großer Mehrwert erwartet, weshalb am Ende jeder positiv überrascht von dem Ergebnis war. Hauptsächlich die ersten beiden Tage haben dem Team vermutlich sehr viel Arbeit und zukünftige Diskussionen erspart. Sehr positiv anzumerken ist, dass jeder Beteiligte sehr konzentriert mitgearbeitet hat und mit großem Arbeitswillen bei der Sache war. Das führte dazu, dass das Team während der ganzen 5 Tage in einem Flow war und dementsprechend sehr gute Arbeit und konstruktive Beiträge geleistet hat. Allerdings ist auch aufgefallen, dass der Prozess, wie vorgesehen, nicht sinnvoll für ein komplett neues Produkt ist. Dies beinhaltet zu viele Einzelteile und zu viele Details, die das ganze Team zusammen abklären muss. Durchaus sinnvoll wäre das für eine Erweiterung eines bestehenden Produktes. In der abgewandelten Variante dieses Teams wurden trotzdem viel mehr Ergebnisse erzielt als erwartet. In nur fünf Tagen konnte ein Konzept erstellt werden, alle Screens des Produktes gezeichnet werden und sich auf ein Logo und einen Brand-Name geeinigt werden. Da alle Teammitglieder hauptberuflich andere Dinge machen, wie ein Studium oder Vollzeitarbeit, hätte es ohne den Sprint wohl sehr lange gedauert, bis diese Dinge geklärt worden wären.

Diesbezüglich war sich das Team auch einig, einen Follow-Up Sprint zu starten, sobald das Grundprodukt implementiert ist und erste User-Tests gemacht wurden.