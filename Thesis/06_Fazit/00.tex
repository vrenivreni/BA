\chapter{Schlusswort}
\section{Zusammenfassung}
Nachdem sich Agrishare im Rahmen des Hackaburgs bildet, steht bereits ein grobes Konzept. Demnach soll eine Online-Plattform die Ressourcenauslastung in der Agrarbranche effizienter gestalten. Nach dem Erfolg bei dem Event beschließt das Team, ein Startup zu gründen, um die Plattform auf den Markt zu bringen. Nach gründlicher Recherche zur Gründung eines IT-Unternehmens im Allgemeinen, stößt die Gruppe auf eine Reihe erfolgsversprechender Konzepte. So auch die in Kapitel \ref{sec:Grundlagen} beschriebenen Prozesse. 

Aufbauend auf die \textit{Innovator's Method} in Abschnitt \ref{sec:TheInnovatorsMethod} bekommt die Gruppe bereits wertvolle Einblicke in die Kundensicht. So wird klar, dass in der Agrarbranche ein Weg, schnell und effizient Ressourcen zu vermitteln, benötigt wird. Mit diesem Wissen startet das Team einen fünftägigen Workshop nach dem Prinzip des Sprint von \citeauthor{Sprint}, welcher in Abschnitt \ref{sec:Sprint-Grundlagen} zusammengefasst erlärt ist. Die Umsetzung dessen ist in Abschnitt \ref{sec:Sprint-Umsetzung} genauer beschrieben. Allerdings wird der Prozess etwa ab der Hälfte dahingehend abgewandelt, dass die Entwicklung des Prototypen verlängert wird. Der Grund dafür ist, dass dies den ersten Prototypen auf Webbasis darstellt. Daher ist es wichtig, die einzelnen Bestandteile einheitlich aufzubauen, was nicht innerhalb der vorgegebenen Zeit realisierbar ist. Trotzdem schafft es das Team, durch den Sprint viele Details im Vorhinein abzuklären. So können die unterschiedlichen Ansprüche der Mitglieder an das Produkt, welche überraschenderweise sehr voneinander abweichen, miteinander vereinbart werden. Außerdem werden wichtige Begriffe eindeutig definiert, sodass diese zukünftig nicht für Verwirrungen im Team sorgen. Viele andere Punkte, welche unter Abschnitt \ref{sec:Sprint-Umsetzung} genauer erklärt sind, sorgen für ein durchgehend positives Feedback des Teams. Der Workshop wird generell als sehr hilfreich angesehen, obwohl er in diesem Fall abgeändert werden muss. 

Mit den Erkenntnissen aus dem Sprint und den anschließenden Tests können Konzepte der \ac{TLS}-Methode angewandt werden. Der Prototyp aus dem Sprint-Prozess definiert die Grundlage, womit das \textit{Innovation Accounting} Konzept in Abschnitt \ref{sec:LeanStartup-InnovationAccounting} grundlegend beginnt. Allerdings fällt bereits in den ersten Nutzer-Tests auf, dass die tatsächliche Problematik ausschließlich in der Vermittlung der Ressourcen liegt. Die ursprüngliche Hypothese, nach welcher zusätzlich eine reibungslose Transaktion ermöglicht werden soll, wird von den Kunden widerlegt. An dieser Stelle wird zum ersten Mal das Prinzip des \textit{Validated Learning} angewandt, welches in Abschnitt \ref{sec:LeanStartup-ValidatedLearning} erklärt wird. Daraus resultiert die Entscheidung ein \textit{Pivot} durchzuführen. So wird die Plattform lediglich für die Vermittlung entwickelt, wie in Abschnitt \ref{sec:LeanStartup-Pivot} eingehend beschrieben ist. Außerdem zeigt sich in einer weiteren Iteration der \ac{BML} Schleife, dass Vorschläge für attraktive Angebote auf der Startseite positiv auf die Endnutzer wirken. Daraus entsteht mittels der in Abschnitt \ref{sec:LeanStartup} beschriebenen Methoden ein strukturiertes Konzept. 

Damit ist Schritt drei der in Abschnitt \ref{sec:TheInnovatorsMethod} auf Abbildung \ref{fig:TheInnovatorsMethod} dargestellten Konzeptfolge abgeschlossen. So kann auf diesem Ergebnis aufbauend ein Geschäftsmodell entwickelt werden. Dieses wird nach dem in der \textit{Innovator's Method} vorgeschlagenem Prinzip der \ac{BMC} abgebildet. Ein erster Entwurf dieser ist in Abbildung \ref{fig:BMC_Structure} dargestellt. Diese wird nach dem in Abschnitt \ref{BMC_Kapitel} beschriebenen Aufbau erstellt. Im weiteren Verlauf des Projektes wird die Canvas voraussichtlich weiter angepasst.

\section{Fazit und Ausblick}

\begin{table}[]
\caption{Aufstellung getätigter Suchanfragen zur Prozessevaluierung}
\label{tab:Search}
\begin{tabular}{lll}
\hline
\multicolumn{1}{|l|}{\textbf{Suchbegriff}}                                                                  & \multicolumn{1}{l|}{\textbf{Suchmaschine}} & \multicolumn{1}{l|}{\textbf{Ergebnisse}}                                         \\ \hline
\multicolumn{1}{|l|}{\multirow{3}{*}{\begin{tabular}[c]{@{}l@{}}User-Centered \\ Design\end{tabular}}}      & \multicolumn{1}{l|}{Google Scholar}        & \multicolumn{1}{l|}{\multirow{3}{*}{veraltete Ansätze (um 1990)}}                \\ \cline{2-2}
\multicolumn{1}{|l|}{}                                                                                      & \multicolumn{1}{l|}{Link Springer}         & \multicolumn{1}{l|}{}                                                            \\ \cline{2-2}
\multicolumn{1}{|l|}{}                                                                                      & \multicolumn{1}{l|}{Cite Seer X}           & \multicolumn{1}{l|}{}                                                            \\ \hline
\multicolumn{1}{|l|}{\multirow{3}{*}{Lean Startup}}                                                         & \multicolumn{1}{l|}{Google Scholar}        & \multicolumn{1}{l|}{Lean Startup Ansätze angelehnt an den Originalprozess}       \\ \cline{2-3} 
\multicolumn{1}{|l|}{}                                                                                      & \multicolumn{1}{l|}{Link Springer}         & \multicolumn{1}{l|}{Lean Startup Konzepte}                                       \\ \cline{2-3} 
\multicolumn{1}{|l|}{}                                                                                      & \multicolumn{1}{l|}{Cite Seer X}           & \multicolumn{1}{l|}{Anwendungsbeispiele ohne Evaluierung}                        \\ \hline
\multicolumn{1}{|l|}{\multirow{3}{*}{\begin{tabular}[c]{@{}l@{}}Lean Startup \\ Success Rate\end{tabular}}} & \multicolumn{1}{l|}{Google Scholar}        & \multicolumn{1}{l|}{Lean Startup Ansätze angelehnt an den Originalprozess}       \\ \cline{2-3} 
\multicolumn{1}{|l|}{}                                                                                      & \multicolumn{1}{l|}{Link Springer}         & \multicolumn{1}{l|}{Schwierigkeiten in etablierten Unternehmen}                  \\ \cline{2-3} 
\multicolumn{1}{|l|}{}                                                                                      & \multicolumn{1}{l|}{Cite Seer X}           & \multicolumn{1}{l|}{Anwendungsbeispiele ohne Evaluierung}                        \\ \hline
\multicolumn{1}{|l|}{\multirow{3}{*}{\begin{tabular}[c]{@{}l@{}}Sprint Google \\ Ventures\end{tabular}}}    & \multicolumn{1}{l|}{Google Scholar}        & \multicolumn{1}{l|}{\multirow{2}{*}{Anwendungsbeispiele}}                        \\ \cline{2-2}
\multicolumn{1}{|l|}{}                                                                                      & \multicolumn{1}{l|}{Link Springer}         & \multicolumn{1}{l|}{}                                                            \\ \cline{2-3} 
\multicolumn{1}{|l|}{}                                                                                      & \multicolumn{1}{l|}{Cite Seer X}           & \multicolumn{1}{l|}{Veraltete themenfremde Ergebnisse}                           \\ \hline
\multicolumn{1}{|l|}{\multirow{3}{*}{\begin{tabular}[c]{@{}l@{}}Lean Process \\ Evaluation\end{tabular}}}   & \multicolumn{1}{l|}{Google Scholar}        & \multicolumn{1}{l|}{\multirow{2}{*}{Lean Anwendungsbeispiele in der Produktion}} \\ \cline{2-2}
\multicolumn{1}{|l|}{}                                                                                      & \multicolumn{1}{l|}{Link Springer}         & \multicolumn{1}{l|}{}                                                            \\ \cline{2-3} 
\multicolumn{1}{|l|}{}                                                                                      & \multicolumn{1}{l|}{Cite Seer X}           & \multicolumn{1}{l|}{Veraltete themenfremde Ergebnisse}                           \\ \hline                                                                           
\end{tabular}
\end{table}

Es ist sehr komplex, den Erfolg der in dieser Arbeit vorgestellten Konzepte mit Zahlen zu belegen. Auch eine ausführliche Online-Recherche liefert hier keine Ergebnisse. Die in Tabelle \ref{tab:Search} dargestellten Suchanfragen stellen die naheliegendsten Suchbegriffe zu diesem Thema dar. Dabei fällt auf, dass diese in den meisten Fällen weitere Ansätze liefern, welche an die Originalprozesse aus Kapitel \ref{sec:Grundlagen} angelehnt sind. Darüber hinaus können einige Anwendungsbeispiele gefunden werden. So beschreibt \citeA{peltola2017adapting} das Sprint-Konzept von \citeauthor{Sprint} als durchaus sinnvoll, nachdem dieser die Methode zusammen mit einem Virtual Reality Kurs testete. Der Autor betont hier die positiven Ergebnisse, die im Rahmen des Workshops entstanden sind. Außerdem stellt \citeA{SlowFastDesign} \textit{schnelle} und \textit{langsame} Designprozesse gegenüber. Die im Rahmen dieser Arbeit bearbeiteten Prozesse sind hier in die Kategorie der \textit{schnellen} Prozesse einzuordnen, da sie durch enge Zusammenarbeit mit der Zielgruppe für schnelle erste Ergebnisse sorgen sollen. Im Gegensatz dazu stehen Konzepte, bei welchen im Vorfeld das Gesamtprodukt detailgenau geplant wird, was als \textit{langsamer} Design-Ansatz angesehen wird. Das Resultat aus dieser Gegenüberstellung ist, dass beide Ansätze gleichwertig sinnvoll sind. Beide Grundkonzepte bieten sowohl Vor-, als auch Nachteile, daher sollte abhängig vom Projekt entschieden werden, welche Methode angewandt wird. Allerdings werden \textit{\ac{UCD}} Konzepte bereits lange vor der Entwicklung von \textit{\ac{TLS}} und \textit{Sprint} untersucht. Darunter versteht man eine Grundlage für Lean Design dahingehend, dass der Endnutzer verstärkt in die Entwicklung einbezogen wird. So haben \citeauthor{vredenburg2002survey} sowohl \citeyear{vredenburg2002survey}, als auch \citeyear{mao2005state} belegt, dass dieser Ansatz gewinnbringendere Endprodukte hervorbringt, da diese höhere Benutzerfreundlichkeit und generell einen höheren Wert für den Kunden aufweisen. Außerdem werden bei der Recherche viele Schwierigkeiten der Einführung von Lean Methoden in etablierte Unternehmen gefunden, wobei es sich hauptsächlich um die Änderung interner Strukturen handelt. Da dies nicht auf ein Startup zutrifft, werden diese Schwierigkeiten für die Arbeit als irrelevant betrachtet.

Daher kann keine eindeutige Aussage darüber getroffen werden, ob die oben beschriebenen Prozesse sinnvoll sind. Diese Fragestellung zu beantworten, gestaltet sich grundsätzlich äußerst komplex, da es schwierig ist, Erfolg von Startups anhand der angewandten Prozesse zu beziffern. Denn, scheitert ein Startup, kann das unterschiedliche Gründe haben. Das ist nicht zwingend abhängig davon, welche Konzepte bei der Gründung angewandt werden. Darüber hinaus ist der Erfolg eines Unternehmens auch nicht ausschließlich auf intern angewandte Prozesse zurückzuführen. So wird ein Produkt, wofür es keinerlei Nachfrage auf dem Markt gibt, unabhängig von Konzepten oder Marketingstrategien scheitern. Daher ist ein Grundgedanke der Lean-Methoden, die Nachfrage durch die Kundenorientierung im Vorhinein zu prüfen. Damit soll vorab verhindert werden, dass ein Unternehmen gegründet wird, wessen Produkt nicht angenommen wird. Wird ein Prototyp in den Usertests durchgehend schlecht bewertet, fällt idealerweise bereits zu Beginn des Startups auf, dass für das Produkt kein Markt existiert und somit eine Unternehmensgründung wahrscheinlich fehlschlagen wird. Im Umkehrschluss sollten dabei die Scheiterungsstatistiken von Startups zurückgehen, was nicht der Fall ist. Dabei stellt sich die Frage, wie viele von den gescheiterten Startups Lean Methoden angewandt haben. Auch dafür können keine konkreten Zahlen angegeben werden. Ein junges Unternehmen, welches trotz der Anwendung dieser Methoden scheitert, könnte die Konzepte unter Umständen auch unvollständig und inkonsequent angewandt haben. Dann würden diese fälschlicherweise die Erfolsrate der Lean-Konzepte verschlechtern. 

Was hingegen im kritischen Umgang mit modernen Design-Prozessen auffällt, ist, dass der größte Nachteil derer die aufzuwendende Zeit ist. Denn das Team stimmt zuvor das Design des Produktes mit der Zielgruppe ab, anstatt sofort mit der Implementierung zu beginnen. Bekommt ein Team also im Rahmen der einzelnen Iterationen keine neuen Einblicke in die Bedürfnisse der Kunden, können diese als unnötig angesehen werden. Jedoch scheint es utopisch, ein von Anfang an perfekt auf den Kunden abgestimmtes Produktkonzept zu haben, welches sich auch in Usertests als ideal erweist. Trotzdem ist es nicht unwahrscheinlich, dass sich eine gute Idee auch ohne bekannte Lean-Prozesse vermarkten lässt. Das Agrishare-Team, sowie zahlreiche weitere Beispiele der Autoren oben beschriebener Konzepte zeigen exemplarisch, dass die Abstimmung auf die Zielgruppe in den meisten Fällen sinnvoll ist. So hätte Agrishare vermutlich viel Zeit in die Umsetzung der Transaktion investiert, welche sich als unnötig herausgestellt hat. Durch die \ac{TLS}-Methoden konnte dies zuvor eliminiert werden. So wird im Ergebnis Implementierungszeit und -aufwand gespart durch die Anwendung des Lean Konzeptes. Ähnliche Ergebnisse zeigen sich auch an den Beispielen der Sprint- und \ac{TLS}-Methoden. Daher lässt sich grundlegend feststellen, dass die Design-Prozesse durchaus sinnvoll sind.

Im weiteren Verlauf der Startup-Gründung am Beispiel Agrishare beschließt das Team, die getesteten Konzepte weiterhin umzusetzen und in die weitere Unternehmensstruktur einzubauen. Da die Plattform derzeit noch nicht auf dem Markt ist, lassen sich noch keine Aussagen über Wachstum oder Erfolg machen. Das Team ist allerdings sehr zuversichtlich, mithilfe der Kundennähe langfristigen Erfolg mit Agrishare zu haben.