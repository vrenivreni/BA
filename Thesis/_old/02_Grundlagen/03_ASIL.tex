\section{\acl{ASIL}}\label{sec:GrundlagenASIL}

Mittels \acf{ASIL} wird der Sicherheitsgrad einer Anwendung definiert. Dieser wurde unter der ISO 26262 standardisiert und hat sich bereits im Automotiv Bereich etabliert. Hier werden die verschiedenen \acs{ASIL}-Level verwendet, um das Risiko einer Funktion mit deren Sicherheitsansprüchen in Relation zu stellen. Es wird lediglich das Risiko für den Fahrer und sein Umfeld berechnet. 
Das Level wird unter Zuhilfenahme dreier Parametern definiert:
\begin{itemize}
	\item \textbf{Severity - S} 
	\newline Schwere des Fehlers, Grad der möglichen Verletzungen
	\subitem S0: keine Verletzungen
	\subitem S1: leicht bis mittelschwere Verletzungen
	\subitem S2: schwere Verletzungen, Überleben wahrscheinlich
	\subitem S3: schwerste Verletzungen, Überleben unwahrscheinlich
	\item \textbf{Exposure - E}
	\newline Eintrittswahrscheinlichkeit des Fehlers
	\subitem E1: seltenes Auftreten
	\subitem E2: gelegentliches Auftreten
	\subitem E3: häufiges Auftreten
	\subitem E4: ständiges Auftreten
	
	\newpage	
	\item \textbf{Controllability - C}
	\newline Beherrschbarkeit des Fehlers
	\subitem C0: sichere Beherrschung
	\subitem C1: einfache Beherrschbarkeit
	\subitem C2: normale Beherrschbarkeit
	\subitem C3: schwierige Beherrschbarkeit
\end{itemize}
Anhand der Grafik wird die ASIL-Klassifikation bestimmt.
\singlefigure{\label{fig:ASILLevel}}{99_IMG/03_manifest/ASILLevel.png}{ASIL Klassifikation}

Alles was sich im Bereich \emph{QM} befindet, muss nicht extra abgesichert werden, da es in der Regel mit den Grundstandards abgedeckt ist. Die andern Level erläutern sich wie folgt:
\begin{itemize}
	\item \textbf{ASIL A}
	\newline Empfohlene Ausfallwahrscheinlichkeit kleiner 10$^{-6}$ pro Stunde
	\item \textbf{ASIL B}
	\newline Empfohlene Ausfallwahrscheinlichkeit kleiner 10$^{-7}$ pro Stunde
	\item \textbf{ASIL C}
	\newline Geforderte Ausfallwahrscheinlichkeit kleiner 10$^{-7}$ pro Stunde
	\item \textbf{ASIL D}
	\newline Geforderte Ausfallwahrscheinlichkeit kleiner 10$^{-8}$ pro Stunde 
\end{itemize}
