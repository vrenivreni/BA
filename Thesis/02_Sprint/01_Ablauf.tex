\section{Ablaufbeschreibung}
5 Tage

\subsection*{Tag 1}
\paragraph{Long Term Goal}
Der Sprint beginnt mit der Festlegung eines Langzeitzieles. Darunter versteht man einen Satz, der aussagt, was mit dem Projekt auf lange Sicht erreicht werden soll. In einer Gruppendiskussion sollen hier Fragen wie \textit{Warum wird das Projekt durchgeführt? Wo sehen wir uns in 6 Monaten, in einem Jahr oder sogar in fünf Jahren?} geklärt werden. Oft kann es hier dazu kommen, dass die Teammitglieder unterschiedliche Erwartungen haben. Diese Aufgabe ist also einerseits dafür gedacht, die Erwartungen des Gesamtteams zu einer realistischen Gesamterwartung zusammenzufassen. Andererseits soll dieses Langzeitziel auch während des gesamten Sprints dazu dienen, den Fokus nicht zu verlieren. Da man sich diesen Satz immer wieder ins Gedächtnis ruft, ist es einfacher, fokussiert zu bleiben. Damit dieser Satz den kompletten Sprint über sichtbar ist, wird er gut sichtbar auf ein Whiteboard notiert. 

\paragraph{Sprint-Fragen}
Nachdem die Erwatungen aller Teilnehmer zusammengefasst und geklärt wurden, werden im nächsten Schritt alle Bedenken gesammelt. Dafür werden offene Fragen aus der Runde auf ein weiteres Whiteboard geschrieben. Jeder Teilnehmer hat nun die Aufgabe, seine Bedenken oder Ängste zu einer Frage zu formulieren oder auch Fragen, die er gerne im Laufe der Zeit beantwortet haben möchte, zu stellen. Hierbei wird nicht aussortiert oder geurteilt, alle Beiträge aus dem Team werden notiert und somit auch über den Sprint-Zeitraum im Gedächtnis behalten. 

\paragraph{Map}
Die nächste Aufgabe besteht darin, das Gesamtprojekt auf einer Karte oder einem Graphen festzuhalten. Dieser Graph soll das Projekt möglichst einfach aber trotzdem verständlich darstellen. Festgehalten wird diese Karte auf dem ersten Whiteboard direkt unter dem Long Term Goal. Dabei wird mit den Schlüsselpersonen begonnen, welche auf der linken Seite untereinander gelistet werden. Darunter versteht man meist unterschiedliche Kundengruppen oder auch andere wichtige Personengruppen, wie beispielsweise der Staat. Danach wird rechts das Ende des Prozesses aufgeschrieben. Das soll aus einem Stichwort bestehen, welches das in gewisser Weise auch den Sinn des ganzen Projekts beschreibt, wie zum Beispiel \textit{Behandlung} oder \textit{Kauf}. Nachdem Start und Ziel des Prozesses festgehalten sind, sollen nun die Zwischenschritte eingetragen werden. Das beinhaltet jeden Schritt, den eine Schlüsselperson durchlaufen muss, um an das Ziel rechts zu kommen. Sehr wichtig ist es, diese Schritte zwar sehr abstrakt zu halten, aber trotzdem die wichtigsten Schritte zu erkennen und aufzuschreiben. Als Richtwert sollten am Ende zwischen fünf und 15 Schritte dargestellt sein.

Nach dieser Aufgabe sollte es bereits ca. 11:00 Uhr sein. Daher wird empfohlen, eine einstündige Mittagspause einzulegen.

\paragraph{Ask the Experts}
Für diese Aufgabe sollte man im Vorfeld Personen einladen, welche sich in verschiedenen Teilbereichen des Gesamtprojekts besonders gut auskennen. Diese können auch aus dem Sprint-Team selbst sein oder extern für diesen Nachmittag dazustoßen. Das hat den Vorteil, dass sich selbst die Projektleiter nicht in jedem Teilgebiet gut genug auskennen, um detaillierte Beschreibungen und Erklärungen zu liefern. Welche Experten man einladen sollte ist natürlich immer abhängig von dem Projekt. Allerdings gibt es grobe Vorschläge von Google Ventures.
Die Gebiete, aus denen man idealerweise einen Fachmann zu Rate ziehen sollte, sind folgende: 
\begin{itemize}
	\item \textbf{Strategie:}
	Es wird hier durchaus empfohlen, jemanden, bespielsweise den Decider, über die Gesamtstrategie vortragen zu lassen. Das dient dem Allgemeinverständnis des Teams über das Ziel und die Schritte zum Erreichen dieses Ziels. 
	\item \textbf{Kundensicht:}
	Diese Person sollte ein Mitarbeiter sein, welcher am meisten von der Kundenseite versteht. Dieser sollte auch in der Lage sein, die Sichtweise der Kunden präzise zu erklären und mögliche Risiken oder interessante Einsichten aufbringen.
	\item \textbf{Produktexperte:}
	Es ist außerdem wichtig, jemanden einzuladen, der die Produktseite vertritt, im Gegensatz zur Kundenseite. Dieser sollte die Mechanik des Endproduktes kennen und über Einzelheiten in der Produktion Bescheid wissen. Da es in diesem Feld sehr viele Einzelbereiche gibt, wie zum Beispiel Finanzen, Tech/Logistik oder Marketing, ist es üblich, mehrere Experten aus den einzelnen Bereichen einzuladen. Hier geht es vor allem darum, herauszufinden, wie diese Teile zusammenpassen können. Die Art und Anzahl der Experten hängt wiederum vom Projekt ab und kann deshalb nicht pauschal eingegrenzt werden.
	\item \textbf{Probleme in der Vergangenheit:}
	Eventuell gab es in der Vergangenheit bereits Personen oder Teams, welche sich intensiver mit dem Thema befasst haben. Dann könnten diese darüber berichten, welche Probleme aufgekommen sind oder ob es bereits Lösungsansätze gibt.
\end{itemize}

Um den zeitlichen Rahmen des Workshops einzuhalten, sollte man diese Expertenrunden auf 30 Minuten pro Person eingrenzen. Falls der Fachmann nicht Teil des Sprint-Teams ist, sollte zuerst der Grund für den Sprint erläutert werden und die Whiteboards mit dem Sprint-Ziel, der Map und den Fragen kurz nähergebracht werden. Danach sollte der Experte frei über sein Spezialgebiet im Bezug auf das Projekt erzählen. Das Team sollte dabei viele Fragen stellen, sodass jedes Mitglied ein möglichst tiefes Verständnis für diesen Bereich bekommt. Falls nötig, werden nun die Whiteboards abgeändert. Das heißt, falls sich Änderung auftun, welche nicht mit dem Ziel oder der Map vereinbar sind, dürfen diese an der Stelle angepasst werden. Außerdem sollten wichtige Fragen bei den Sprint-Fragen hinzugefügt werden.

Während der Expertenrunde soll das gesamte Team nicht nur Fragen stellen, sondern auch Notizen machen, jede Person separat. Dafür wird eine Vorgehensweise vorgeschlagen, die \textit{How Might We} genannt wird. Dabei bekommt jede Person einen Block mit Haftnotizen und einen schwarzen Marker. In die linke obere Ecke werden die Buchstaben HMW geschrieben, um den Fragesatz einzuleiten. Wenn man etwas interessantes hört, wird diese Information als Frage formuliert auf die Haftnotiz geschrieben und diese Notiz beiseite gelegt.

\paragraph{Notizen sortieren}
Nachdem die Expertenrunden abgeschlossen sind, werden die Haftnotizen von allen Teammitgliedern ungeordnet an eine freie Wand geklebt. Dann ordnet das Team die Notizen zu logischen Gruppen und findet eine passende Überschrift für die Kategorie. Typischerweise können die restlichen Notizen ohne logischen Zusammenhang zu der Gruppe \textit{Sonstige} zusammengefasst werden.

\paragraph{Notizen bewerten}
Die geordneten Notizen sollen nun priorisiert werden. Dafür bekommt jedes Teammitglied zwei Sticker und der Decider vier. Nachdem sich alle das Gesamtziel und die Fragen erneut vor Augen gerufen haben, sollen sie die Sticker still auf jene Notizen kleben, welche sie für am wichtigsten erachten. Dabei darf die eigene Notiz gewählt werden und es dürfen auf eine Notiz auch mehrere Sticker geklebt werden. Nach der stillen Abstimmung werden die Notizen mit den meisten Stickern zu den logisch passenden Schritten auf der Map geklebt.

\paragraph{Fokus des Sprints}
Zum Abschluss des ersten Sprint-Tages wird der Fokus für die restlichen Tage festgelegt. Dieser Fokus wird alleine vom Decider beschlossen, allerdings darf er das Team um Hilfe bitten. Dafür schreibt jedes Teammitglieder die Schritte der Map auf, die für ihn am wichtigsten sind. Die Vorschläge werden daraufhin auf einem Whiteboard gesammelt und kurz diskutiert, falls es stark abweichende Meinungen gibt. Da dies genug Input für den Decider sein soll, muss dieser nun auf der Map eine Zielgruppe und einen Zielschritt einkreisen. Danach werden wiederum die Sprint-Fragen wiederholt und jede markiert, die nach der Fokusauswahl im Rahmen des Sprints als lösbar scheinen.

\subsection*{Tag 2}
Der Vormittag des zweiten Tages läuft unter dem Motto \textit{Remix and Improve}. Es geht darum, bewährte Methoden oder Teile anderer Produkte auf das Projekt zuzuschneiden oder anzupassen. Wichtig dabei ist es, auch innerhalb anderer Branchen zu suchen und kreativ zu sein. Meist ist es nicht auf den ersten Blick erkennbar, wie andere Produkte auf das eigene Projekt angepasst werden sollen. Oft ist es aber nicht nötig, das Rad neu zu erfinden, da man bereits existierende Einzelteile einfach passend einbauen muss.
\paragraph{Lighting Demos}

\paragraph{Aufgabenverteilung}

\paragraph{Sketch}

\subsection*{Tag 3}

\subsection*{Tag 4}

\subsection*{Tag 5}

\subsection*{Resultat}
